%%%Architettura alto livello%%%
\subsection{Architettura ad alto livello}
Nella seguente sezione verr\`{a} illustrata l'architettura ad alto livello del sistema sviluppato, 
escludendo i dettagli implementativi e legati al linguaggio.
% Componenti del sistema
\subsubsection{Componenti di sistema}
Le principali componenti del sistema sono:
\begin{description}
\item{\textbf{Competition}}\\
La \emph{Competition} \`{e} l'unit\`{a} atta ad orchestrare l'avvio e la conclusione della corsa. Tale componente, dunque, \`{e} stata concepita per 
offrire le seguenti funzionalit\`{a}:
\begin{itemize}
\item Configurazione parametri di gara:
	\begin{itemize}
		\item numero di giri;
		\item numero di concorrenti;
		\item circuito;
	\end{itemize}
\item Gestione della sessione di iscrizione e accettazione concorrenti (configurati a livello della componente \emph{Box})
\item Avvio delle componenti necessarie al monitoraggio della gara (quali ad esempio \emph{Monitor})
\item Avvio controllato della competizione vera e propria nel momento in cui tutti i prerequisiti di inizio sono soddisfatti, ovvero:
\begin{itemize}
\item la competizione \`{e} stata configurata correttamente;
\item le componenti di controllo e gestione della competizione sono attive e in attesa di comandi;
\item i concorrenti sono stati correttamente registrati alla competizione e in attesa di partire;
\end{itemize}
\end{itemize}
\item{\textbf{Competitor}}\\
Il \emph{Competitor} \`{e} l'entit\`{a} pensata ad svolgere la gara. \`{E} caratterizzato dalle seguenti sotto-componenti:
\begin{itemize}
\item \textbf{Auto}, ovvero tutte le caratteristiche fisiche legate all'auto, ovvero:
	\begin{itemize}
		\item motore;
		\item capacit\`{a} del serbatoio;
		\item massima accelerazione;
		\item massima velocit\`{a};
		\item gomme montate (mescola, modello, tipo);
		\item livello usura gomme;
		\item livello della benzina nel serbatoio;
	\end{itemize}
\item \textbf{Guidatore}, cio\`{e} le informazioni che descrivono pi\`{u} dettagliatamente il concorrente in gara:
	\begin{itemize}
		\item nome e cognome pilota;
		\item nome scuderia
	\end{itemize}
\item \textbf{Strategia}, ovvero la strategia che sta adottando il pilota, suggerita dai box e dinamica nel corso della gara:
	\begin{itemize}
		\item stile di guida, variabile tra conservativo, normale, aggressivo, a seconda dello stato della macchina e delle
			previsioni fatte dai box
		\item numero di lap prima del pit stop
		\item addizionalmente, quando viene fatto un pit stop, la strategia determina anche quali siano le nuove gomme da montare,
		 	la quantit\`{a} di benzina da avere nel serbatoio e il tempo impiegato per il pit stop.
	\end{itemize}
\end{itemize}
Tutte queste informazioni insieme creano quello che viene definito il concorrente di gara. 
Tali informazioni verranno poi usate nel corso della gara per:
	\begin{itemize}
		\item scegliere al momento giusto la miglior traiettoria da seguire, in base alla presenza o meno di altri concorrenti
			nelle vicinanze e alla difficolt\`{a} del tratto;
		\item fornire costantemente aggiornamenti sul suo stato (tramite una parte del modulo \emph{Stats}, informando il computer di bordo
			riguardo a:
			\begin{itemize}
				\item livello di usura gomme;
				\item livello di benzina;
				\item checkpoint attraversato con tempo di arrivo;
				\item insieme al checkpoint verranno aggiornate le informazioni relative a settore e lap;
				\item velocit\`{a} massima raggiunta.
			\end{itemize}
		\item contattare ad ogni giro il box per ottenere una strategia aggiornata;
		\item se suggerito dai box, effettuare un pitstop;
		\item ritirarsi dalla gara una volta che le condizione dell'auto non permettano di poter correre ulteriormente;
		\item banalmente, continuare a correre fino alla fine delle lap prestabilite, dopodich\`{e} fermarsi.
	\end{itemize}
\item{\textbf{Circuit}}\\
Il circuito \`{e} una risorsa finalizzata ad offrire il piano su cui svolgere la competizione. \`{E} condivisa fra tutti i concorrenti in gara e
offre un insieme di funzionalit\`{a} per poter conoscere le caratteristiche dei vari tratti della pista (compresi i concorrenti presenti
al momento dell'attraversamento). \`{E} composto dalle seguenti sottocomponenti:
\begin{itemize}
\item \textbf{Checkpoint}:
i \emph{Checkpoint} rappresentano punti di arrivo intermedi del circuito. Come una suddivisione in fette da 1 secondo possono discretizzare 1 minuto,
cos\`{i} i \emph{Checkpoint} discretizzano il circuito. Ogni \emph{Checkpoint} introduce un tratto della pista potenzialmente diverso da quello precedente.
Per esempio, il $\emph{Checkpoint}_n$ potrebbe essere il punto di entrata di un tratto della pista accessibile ad un numero massimo di 4 concorrenti insieme,
mentre il successivo $\emph{Checkpoint}_{n+1}$ potrebbe esporre un tratto pi`{u} stretto e quindi accessibile solo a 2 concorrenti. Schematizzando, 
il \emph{Checkpoint} \`{e} caratterizzato da:
\begin{itemize}
\item \textbf{molteplicit\`{a}}: ovvero il numero di concorrenti che possono trovarsi contemporaneamente nel tratto a seguire;
\item \textbf{posizione nella pista}: un \emph{Checkpoint} pu\`{u} essere il traguardo, l'inizio del settore, la fine di un settore, all'uscita dei box, l'entrata
					per i box, i box oppure un punto intermedio fra altri due \emph{Checkpoint};
\item \textbf{tempi di arrivo}: ogni \emph{Checkpoint} tiene traccia dell'istante in cui un concorrente ci \`{e} passato sopra.
\end{itemize}
\item \textbf{Path}: rappresenta una delle possibili traiettorie da usare per andare da un \emph{Checkpoint} a quello successivo. La traiettoria presenta
	un numero di \emph{Path} uguale alla molteplicit\`{a} del \emph{Checkpoint} che la precede. Ogni \emph{Path} \`{e} descritto da:
	\begin{itemize}
		\item lunghezza
		\item angolo
		\item grip, ovvero l'aderenza sul tratto
	\end{itemize}
Normalmente i path appartenenti allo stesso tratto differiscono di poco.
\item \textbf{Iteratore}: l'unit\`{a} permette di sapere la struttura della pista. Si suppone venga usata per sapere quale \emph{Checkpoint} ne segue un altro,
	oppure per sapere dove sia il \emph{Checkpoint} di inizio box. 
\end{itemize}
\item{\textbf{Stats}}\\
Questa componente mantiene la storia della gara e offre un insieme di funzionalit\`{a} che permettono di elaborare tali dati per offrirne differenti viste:
	\begin{itemize}
		\item migliori performance in un determinato istante di tempo
		\item classifica aggiornata per istante di tempo
		\item informazioni sui concorrenti relative ad un particolare lap, checkpoint o settore (in una specifica lap);
	\end{itemize}
\item{\textbf{Box}}\\
Il \emph{Box} \`{e} l'entit\`{a} che si occupa di gestire la configurazione e la corsa di un concorrente. Durante la competizione, il \emph{Box} verifica costantemente
lo stato dell'auto e fornisce eventuali cambi di strategia se ritenuto opportuno. Inoltre decide quando i pitstop del concorrente con tutte le caratteristiche 
ad esso legate, quali:
	\begin{itemize}
		\item benzina da aggiungere nel serbatoio
		\item gomme da montare
	\end{itemize}
Ogni \emph{Box} \`{e} caratterizzato da uno fra 4 tipi di strategia, diversi per grado di ``ottimismo'' nelle valutazioni e nei calcoli dati lo stato della 
macchina, le medie calcolate e lo stile di guida del concorrente:
	\begin{enumerate}
		\item \textbf{Cautious}: cauto, sottostima il numero di giri ancora fattibili;
		\item \textbf{Nodmal}: stima abbastanza realistica delle possibilit\`{a} del concorrente, considera anche un margine di errore nei calcoli
			per effettuare una valutazione;
		\item \textbf{Risky}: le stime vengono effettuate in base a calcoli esatti che di solito non tengono in considerazione fattori che nella
			realt\`{a} possono incidere in modo negativo;
		\item \textbf{Fool}: nella realt\`{a} normalmente non si arriva a tanto, ma per fini di test \`{e} stato inserito anche un tipo di strategia
			che sovrastima le possibilit\`{a} del concorrente, portandolo a squalifica quasi certa.
	\end{enumerate}
Ci\`{o} che il box suggerisce al concorrente durante la gara \`{e}:
	\begin{itemize}
		\item stile di guida. Verr\`{a} suggerito uno stile pi\`{u} conservativo se i consumi si sono rivelati maggiori del previsto e viceversa;
		\item numero di lap al pitstop
	\end{itemize}
Il \emph{Box} riceve informazioni sullo stato del concorrente alla fine di ogni settore e ricalcola la strategia alla fine del secondo settore. Il concorrente
richiede la nuova strategia al box in prossimit\`{a} del checkpoint dove \`{e} possibile proguire o andare ai box.\\
\`{E} sembrato pi\`{u} realistica la scelta di non calcolare la strategia alla fine del terzo settore, perch\`{e} si suppone che nella realt\`{a} non si possa essere
cos\`{i} veloci da calcolare una nuova strategia istantaneamente alla fine del circuito con i dati del terzo settore. \`{E} piuttosto pi\`{u} probabile che 
qualunque cambio di strategia o richiesta di rientro ai box venga stabilita gi\`{a} alla fine del secondo settore. In modo che in prossimit\`{a} dei box il concorrente
possa ottenere l'informazione istantaneamente e possa quindi decidere come e dove procedere.
\item{\textbf{Radio}}\\
\item{\textbf{Monitor}}\\
\end{description}
 - lista delle componenti con descrizione ad alto livello del loro scopo
 - se necessario fornire un diagramma delle componenti
% Interazione fra le componenti
\subsubsection{Interazione fra le componenti}
 - descrivere ad alto livello l'interazione fra componenti e se necessario aiutarsi con OCR cards
% Strategia adottata per la correttezza temporale
\subsubsection{Strategia adottata per la correttezza temporale}
% Dimostrazione dell'assenza di stallo
\subsubsection{Assenza di stallo}
%%% Architettura in dettaglio %%
\subsection{Architettura in dettaglio}
% Elenco dei task con descrizione
\subsubsection{Risorse attive}
% Elenco risorse condivise con descrizione
\subsubsection{Risorse passive}
\begin{itemize}
\item{Risorse protette}
\item{Altre risorse}
\end{itemize}
%"Analisi della concorrenza"
\subsubsection{Analisi della concorrenza}
	%. analisi dell'interazione risorse e task (senza menzionare la distribuzione)
	%. dimostrazione assenza di racecondition
	%. dimostrazione assenza di starvation
\begin{itemize}
\item{Interazione tra risorse condivise e task}
\item{Assenza di racecondition}
\item{Assenza di starvation}
\end{itemize}
%"Distribuzione"
\subsubsection{Distribuzione}
	%. Elenco risorse distribuite
	%. Interazione risorse distribuite
	%. Misure di fault tolerance
\begin{itemize}
\item{Componenti distribuite}%Con motivazione
\item{Interazione fra le componenti distribuite}
\item{Misure di fault tolerance}
\end{itemize}
% Inizializzazione gara
\subsection{Inizializzazione competizione}
% Stop gara
\subsection{Stop competizione}
