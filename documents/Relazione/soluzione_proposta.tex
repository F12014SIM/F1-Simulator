%%%Architettura alto livello%%%
\subsection{Architettura ad alto livello}
% Componenti del sistema
\subsubsection{Componenti di sistema}
 - lista delle componenti con descrizione ad alto livello del loro scopo
 - se necessario fornire un diagramma delle componenti
% Interazione fra le componenti
\subsubsection{Interazione fra le componenti}
 - descrivere ad alto livello l'interazione fra componenti e se necessario aiutarsi con OCR cards
% Strategia adottata per la correttezza temporale
\subsubsection{Strategia adottata per la correttezza temporale}
% Dimostrazione dell'assenza di stallo
\subsubsection{Assenza di stallo}
%%% Architettura in dettaglio %%
\subsection{Architettura in dettaglio}
% Elenco dei task con descrizione
\subsubsection{Risorse attive}
% Elenco risorse condivise con descrizione
\subsubsection{Risorse passive}
\begin{itemize}
\item{Risorse protette}
\item{Altre risorse}
\end{itemize}
%"Analisi della concorrenza"
\subsubsection{Analisi della concorrenza}
	%. analisi dell'interazione risorse e task (senza menzionare la distribuzione)
	%. dimostrazione assenza di racecondition
	%. dimostrazione assenza di starvation
\begin{itemize}
\item{Interazione tra risorse condivise e task}
\item{Assenza di racecondition}
\item{Assenza di starvation}
\end{itemize}
%"Distribuzione"
\subsubsection{Distribuzione}
	%. Elenco risorse distribuite
	%. Interazione risorse distribuite
	%. Misure di fault tolerance
\begin{itemize}
\item{Componenti distribuite}%Con motivazione
\item{Interazione fra le componenti distribuite}
\item{Misure di fault tolerance}
\end{itemize}
% Inizializzazione gara
\subsection{Inizializzazione competizione}
% Stop gara
\subsection{Stop competizione}
