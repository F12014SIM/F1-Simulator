\label{problematiche_concorrenza}
% - Predicibilità della gara
% - Corretto svolgimento della gara rispetto agli input forniti, in modo indipendente dalle funzionamento
% delle componenti a basso livello de sistema operativo (scheduler....)
% - Evitare race condition su risorse condivise
% - Evitare stalli del sistema
% - Assicurarsi che vi sia almeno un istante t in cui due o più task eseguano in concorrenza
% - Coerenza temporale
% - Evitare starvation
Nel corso dell'analisi ad alto livello del progetto, sono emerse numerose problematiche legate alla coesistenza nel 
sistema di pi\`{u} \emph{task} concorrenti e di risorse tra di essi condivise. \\
\subsubsection{Percorrenza concorrente della pista}
Come da specifica, il sistema dovr\`{a} prevedere una pista per lo svolgimento della gara. I partecipanti percorreranno quindi i tratti del circuito in modo concorrente e questo pone gi\`{a} numerose problematiche per il corretto svolgimento della competizione.
Si pu\`{o} vedere ogni tratto come una risorsa condivisa fra i concorrenti della gara. Ciascuna risorsa avr\`{a} una motleplicit\`{a} limitata, coerentemente a quanto accade nella realt\`{a}. Questo porta inevitabilmente a due problematiche da affrontare:
\begin{enumerate}
\item I concorrenti dovranno attraversare ogni tratto in modo da non violare i limiti di molteplicit\`{a} imposti. Bisogner\`{a} quindi impedire che ad un istante \textbf{t} dello svolgimento della gara, su un tratto con molteplicit\`{a} \textbf{n} vi siano, contemporaneamente, \textbf{n+1} auto che lo stanno attraversando;
\item Evitare l'effetto ``teletrasporto''. Uno possibile scenario che si potrebbe infatti presentare nel corso della gara potrebbe essere il seguente (in caso di progettazione poco attenta):
Un tratto $Tr_N$ a molteplicit\`{a} \textbf{1} viene attraversato da \textbf{2} auto, \textbf{A} e \textbf{B}. Logicamente, quindi, se \textbf{A} arriva prima di \textbf{B}, all'inizio del tratto $Tr_{N+1}$ l'ordine dovr\`{a} essere mantenuto. Ad alto livello si potrebbe pensare, come soluzione plausibile, di affidare ad \textbf{A} la risorsa $Tr_N$ e, una volta effettuato virtualmente l'attraversamento, farla rilasciare e affidarle il tratto $Tr_{N+1}$, mentre \textbf{B} star\`{a} cercando di ottenere$Tr_N$ e percorrerlo.
Questa soluzione non tiene per\`{o} in considerazione le problematiche legate alla gestione dei processi dallo scheduler. Potrebbe infatti accadere che:
\begin{itemize}
\item \textbf{A} ottiene $Tr_N$ e \textbf{B} rimane in attesa;
\item \textbf{A} rilascia $Tr_N$ e viene prerilasciato dallo scheduler;
\item \textbf{B} ottiene $Tr_N$, lo rilascia e successivamente ottiene $Tr_{N+1}$ prima di essere prerilasciato dallo scheduler;
\item \textbf{A} \`{e} di nuovo attivo ma si trova in una posizione non coerente con quanto previsto: \`{e} avvenuto un sorpasso in una zona non consentita.
\end{itemize}
Da questo emerge il problema della gestione dei processi a livello di sistema operativo. Non \`{e} possibile infatti fare assunzione sulle politiche di ordinamento e esecuzione dello scheduler, poich\`{e} dipende dalle scelte architetturali del sistema operativo sottostante. Di conseguenza \`{e} necessario sviluppare un strategia di svolgimento della gara che preservi la coerenza della competizione indipendentemente dal comportamento, talvolta non prevedibile, dello scheduler.
\item Il problema precedente potrebbe essere raggirato introducendo l'accumulo di risorse. Ovvero prima di rilasciare il tratto $Tr_N$, il concorrente $Tr_{N+1}$ deve aver gi\`{a} ottenuto l'accesso al tratto $Tr_{N+1}$. In questo modo, per\`{o}, si potrebbe presentare una prospettiva di stallo. Infatti, se il numero di tratti \`{e} minore o uguale al numero di concorrenti, potrebbe verificarsi un'attesa circolare potenzialmente infinita sui tratti della pista. Bisogna quindi assicurarsi che non ci siano le condizioni per il verificarsi di stalli.
\end{enumerate}
