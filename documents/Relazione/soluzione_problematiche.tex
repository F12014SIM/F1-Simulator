\label{soluzione_problematiche}
\subsection{Entit\`{a} Circuito}
\subsection{Entit\`{a} Concorrente (Lorenzo)}
L'entità concorrente del progetto mappa la soluzione di alcune problematiche analizzate nei paragrafi precedenti. Si tratta di un'entità attiva, costruita come un task che utilizza le altre componenti per la sua gara. In seguito verranno analizzate le interazioni che questa entità hanno fra di loro e con le entità passive che compongono il circuito.
\subsection{Interazione concorrenti - circuito (Lorenzo - ricordarsi assenza di stallo)}
\subsection{Interazione concorrenti - concorrenti (Lorenzo)}
\subsection{Entità box (Lorenzo)}
     \subsubsection{Interazione con il concorrente (Lorenzo)}
     \subsubsection{Cosa fa (Lorenzo)}
     \subsubsection{Distrbuzione del box (Lorenzo)}
\subsection{Gestione delle statistiche di gara}
     \subsubsection{Dati singolo concorrente}
     \subsubsection{Dati globali di gara}
\subsection{Reperimento istantanea di gara}
\subsection{Interazione utente-sistema (Lorenzo)}
     \subsubsection{Osservazione gara ( lato box, lato tv )(Lorenzo)}
     \subsubsection{Intervento sulla gara ( lato box, lato tv )(Lorenzo)}