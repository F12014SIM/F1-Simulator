\label{soluzione_problematiche}
\subsection{Entit\`{a} Circuito}
Nella soluzione proposta il circuito è un'unità composta da più entità passive ad accesso mutuamente esclusivo. Più precisamente:
\begin{itemize}
\item \textbf{Path}: rappresenta la traiettoria percorsa da un concorrente. Un insieme di \textbf{Path} costituisce un tratto e il loro
numero all'interno del tratto identifica la molteplicità dello stesso. Ogni path è caratterizzato da:
\begin{itemize}
\item lunghezza
\item angolo
\item tenuta
\item istante di liberazione
\end{itemize}
I \textbf{Path} di uno stesso tratto possono essere uguali o differire per caratteristiche, rimanendo comunque entro i limiti globali del tratto
(non si potrà ad esempio avere un tratto lungo 11 km in un tratto lungo 42 m).\\
Per quanto riguarda l'\textbf{istante di liberazione}, esso indica da che istante non ci saranno più concorrenti sul \textbf{Path}.
\item \textbf{Checkpoint}: è una risorsa passiva ad accesso mutuamente esclusivo. Rappresenta la coda di accesso ad un tratto. Ogni posizione
della coda è stata arricchita con delle informazioni che aiutano a gestire l'accesso al tratto:
\begin{itemize}
\item \textbf{istante di arrivo}: l'istante di tempo più ottimista in cui l'auto è prevista arrivare oppure l'istante in cui l'auto realmente
arriva al tratto (in base al valore della flag ``arrivato'', descritta in seguito);
\item \textbf{id concorrente}: l'id del concorrente presente nella posizione della coda;
\item \textbf{flag ``arrivato''}: se valorizzata con \emph{true}, significa che il concorrente sta effettivamente attendendo di accedere al tratto
e che il valore temporale segnato nell'\textbf{istante di arrivo} è l'istante di arrivo effettivo. Altrimenti significa che il concorrente non
è ancora arrivato ma arriverà ad un istante maggiore o uguale a quello segnato nell'\textbf{istante di arrivo}.
\end{itemize}
Ogni \textbf{Checkpoint} inoltre punta ad un insieme di \textbf{Path}.
\item \textbf{Racetrack iterator}: ogni concorrente dispone di un'istanza di questo iteratore per poter navigare il circuito e richiedere
di accedere eventualmente ai box.
\end{itemize}
La struttura è stata pensata principalmente per venire a capo del problema dei sorpassi impossibili.\\
Come visto nella sezione \ref{tempo}, non esiste un orologio assoluto. Ogni concorrente segna sulla
coda il tempo di arrivo in base ai tempi accumulati per l'attraversamento dei tratti precedenti. Quindi si può dire che ogni
concorrente aggiorni un proprio orologio relativo all'andamento della sua corsa. Inoltre ogni concorre segna
un tempo ottimistico di arrivo sui tratti che (salvo squalifica) attraverserà. Tale istante sarà di sicuro maggiore dell'istante segnato
nel \textbf{Checkpoint} in cui il concorrente effettivamente è arrivato. Questo permette ad ogni conccorrente di sapere, quindi, non solo
i dettagli relativi alla sua corsa, ma anche quelli relativi agli altri concorrenti (quando necessario). Di conseguenza, se un concorrente
è interessato ad attraversare un tratto, potrà confrontare il suo tempo di arrivo effettivo con i tempi di arrivo effettivi e previsti
degli altri concorrenti. Nel momento in cui il suo istante di arrivo effettivo risulti cronologicamente il più basso della coda, sarà certo
che, stando al suo istante di tempo e a quello relativo agli altri concorrenti, il suo turno per attraversare sia arrivato. Ovvero che, in 
termini di tempo relativi, il suo istante, essendo il più basso, indichi che è il primo arrivato al tratto in quell'istante, avendo così
il diritto di accesso.\\
Quando un concorrente deve valutare la traiettoria (\textbf{Path}) da attraversare in un tratto, può farlo semplicemente valutando le caratteristiche
del tratto. Le prime tre elencate qualche riga più sopra garantiscono un minimo di realismo fisico. Le caratteristiche fisiche del tratto infatti
influiranno sui consumi e i tempi dell'auto. Il concorrente dovrà scegliere in modo da minimizzare entrambi.\\
L'ultimo parametro è anche oggetto di valutazione in quanto serve all'utente verificare lo stato di occupazione del tratto. Se l'istante
di liberazione segnato è maggiore di quello di accesso al tratto del concorrente, significa che l'attraversamento richiederà, in aggiunta, 
l'attesa che il tratto si liberi. In un contesto reale questo potrebbe significare che un concorrente, prendendo una traiettoria, sia rallentato
dal concorrente più lento davanti. Quando un concorrente calcola il proprio tempo di attraversamento chiaramente dovrà aggiornare l'istante
di liberazione della traiettoria. Questo garantisce il realismo fisico concorrente-concorrenti descritto nella sezione \ref{analisi_realismo_fisico}.
I tempi segnati nelle code invece mettono in pratica, in parte, il concetto di orologio relativo enunciato nella sezione \ref{tempo}.
Maggiori dettagli sull'interazione fra concorrenti e circuito e sull'algoritmo che regola l'attraversamento del circuito da parte dei concorrenti
verranno forniti nelle sezioni seguenti.
\subsection{Entit\`{a} Concorrente (Lorenzo)}
L'entità concorrente del progetto mappa la soluzione di alcune problematiche analizzate nei paragrafi precedenti. Si tratta di un'entità attiva, costruita come un task che utilizza le altre componenti per la sua gara. In seguito verranno analizzate le interazioni che questa entità hanno fra di loro e con le entità passive che compongono il circuito.
\subsection{Interazione concorrenti - circuito (Lorenzo - ricordarsi assenza di stallo)}
\subsection{Interazione concorrenti - concorrenti (Lorenzo)}
\subsection{Entità box (Lorenzo)}
     \subsubsection{Interazione con il concorrente (Lorenzo)}
     \subsubsection{Cosa fa (Lorenzo)}
     \subsubsection{Distrbuzione del box (Lorenzo)}
\subsection{Gestione delle statistiche di gara}
Come analizzato nella sezione \ref{analisi_istantanee}, per poter reperire un'istantanea di gara ad un dato istante di tempo, è necessario
disporre di una storia degli eventi avvenuti nel corso della competizione, strutturati in modo che risulti possibile risalire a qualunque evento
(rilevante) già avvenuto.\\ 
Nel corso delle due seguenti sottosezioni verranno analizzate le soluzioni adottate per immagazinare i dati realtivi ai singoli concorrenti
e calcolare quelli relativi alla gara.
     \subsubsection{Dati singolo concorrente}
     Come visto nel capitolo relativo all'entità \textbf{Circuito}, ogni concorrente aggiorna, ad ogni 
     \textbf{Checkpoint}, un orologio relativo alla sua corsa. Per ognuno si può quindi sapere, per ogni checkpoint passato in un determinato giro,
     l'istante di arrivo.\\
     Per implementare, in parte, la soluzione proposta si è quindi pensato di salvare per ogni concorrente una lista di dati cronologicamente ordinata.
     Ogni posizione della lista mantiene informazioni legate ad un determinato istante di tempo. In questo modo è possibile all'idea discussa nella
     sezione \ref{analisi_istantanee}:\\
     \emph{[...]una volta richiesto lo snapshot di un istante temporale, lo stato di ogni entità (utile all'istantanea) in 
     quell'istante sia disponibile[...]}\\
     Ogni posizione della lista fornisce i seguenti dati:
     \begin{itemize}
     \item Time: l'istante di riferimento delle informazioni;
     \item Checkpoint: il tratto puntato da questo \textbf{Checkpoint} che è stato attraversato (che si è cioè finito di attraversare) 
     all'istante dato;
     \item Sector: il settore di appartenenza del tratto;
     \item Lap: il giro di riferimento;
     \item Gas\_Level: il livello di gas presente nel serbatoio alla fine dell'attraversamento;
     \item Tyre\_Usury: il livello di usura gomme alla fine dell'attraversamento;
     \item BestLapNum: la miglior lap dall'inizio della competizione fino all'istante dato;
     \item BestLaptime: il tempo della miglior lap dall'inizio della competizione fino all'istante dato;
     \item BestSectorTimes: i migliori tempi per ogni settore dall'inizio della competizione;
     \item Max\_Speed: la massima velocità raggiunta dall'inizio della competizione;
     \item PathLength: la lunghezza della traiettoria attraversata.
     \end{itemize}
     Lo stato del concorrente ad un istante \emph{t} si può ottenere navigando la lista fino a trovare la posizione che contiene le informazioni
     dell'istante di tempo esatto (poco probabile). Altrimenti, se il tempo richiesto è compreso in un intervallo limitato dagli istanti
     indicati in due posizioni contigue, lo stato sarà anche compreso fra gli stati forniti dalle due posizioni. Più precisamente:\\
     \begin{itemize}
     \item per quanto riguarda la posizione sul circuito, si può indicare fra quali \textbf{Checkpoint} è il concorrente;
     \item per quanto riguarda i migliori tempi e altri dati statistici, valgono quelli indicati dalla posizione con istante di tempo più basso (è chiaramente
     un'approssimazione);
     \end{itemize}
     Ogni posizione della lista è stata progettata come una risorsa passiva ad accesso controllato con guardia. Vale a dire che la risorsa può
     essere acceduta solo nel momento in cui essa sia valorizzata con dei dati di gara. Fino a tal momento l'entità richiedente dovrà rimanere
     in attesa. Questa soluzione aiuta a mettere in pratica il suggerimento discusso durante l'analisi. Ovvero: se uno stato ad un istante
     \emph{t} non è ancora disponibile, il richiedente dovrà rimanere in attesa. In questo caso, nel momento in cui una qualunque entità attiva
     avanzerà la richiesta di un'istantanea ad un dato istante, finchè tutti i concorrenti non avranno fornito il proprio stato relativo 
     a quell'istante, il riechiedente verrà fatto attendere.
     \subsubsection{Dati globali di gara}
     
\subsection{Reperimento istantanea di gara}
\subsection{Interazione utente-sistema (Lorenzo)}
     \subsubsection{Osservazione gara ( lato box, lato tv )(Lorenzo)}
     \subsubsection{Intervento sulla gara ( lato box, lato tv )(Lorenzo)}