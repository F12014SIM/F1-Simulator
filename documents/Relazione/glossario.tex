\section{Glossario}
\begin{list}{}
\item \textbf{A}\\
\textsc{Auto}: l'entit\`{a} utilizzata dal pilota per correre sul circuito.\\
\item \textbf{B}\\
\textsc{Box}: il box rappresenta il supervisore del concorrente. Calcola nuove strategie in base all'andamento della gara che vengono comunicate
periodicamente al competitor. Decide inoltre il momento adatto per un pitstop (e tutti i dettagli legati ad esso).\\
\item \textbf{C}\\
\textsc{Concorrente}: l'entit\`{a} costituita da pilota e auto che partecipa alla gara correndo sul circuito.\\
\textsc{Competizione}: \`{e} l'insieme dei giri, del circuito e del numero di concorrenti che partecipano alla gara.\\
\textsc{Circuito}: la pista, composta da checkpoint, tratti (e traiettorie) e box.\\
%\item \textbf{D}
%\item \textbf{E}
%\item \textbf{F}
%\item \textbf{G}
%\item \textbf{H}
%\item \textbf{I}
%\item \textbf{J}
%\item \textbf{K}
%\item \textbf{L}\\
\item \textbf{M}\\
\textsc{Molteplicit\`{a} di un tratto}: determina quante auto possano stare contemporaneamente su un tratto.\\
%\item \textbf{N}
%\item \textbf{O}
\item \textbf{P}\\
\textsc{Pilota}: colui che guida la macchina.\\
%\item \textbf{Q}
%\item \textbf{R}
\item \textbf{S}\\
\textsc{Settore}: ve ne sono tre per ogni circuito (come da regolamento) e rappresentano un segmento specifico del circuito.
\textsc{Stile di guida}: caratterizza il modo di guidare del pilota. Uno stile di guida pi\`{u} aggressivo porter\`{a} il pilota
a tenere una velocit\`{a} pi\`{u} alta.\\
\textsc{Strategia}: viene utilizzata per determinare lo stile di guida del concorrente ad ogni giro e per ordinare un pitstop in caso\\
di necessit\`{a}. Addizionalmente, se avviene un pitstop, fornisce informazioni sul tempo di stop e la nuova configurazione dell'auto.\\
\item \textbf{T}\\
\textsc{Tratto}: un segmento della pista a caratterizzato da angolo, lunghezza e molteplicit\`{a}.\\
\textsc{Traiettoria}: rappresenta uno dei possibili percorsi all'interno di un tratto. Ogni tratto ha un numero di traiettorie pari alla
molteplicit\`{a}. Tali traiettorie usualmente differiscono per angolo e lunghezza rimanendo entro i limiti del tratto.\\
%\item \textbf{U}
%\item \textbf{V}
%\item \textbf{X}
%\item \textbf{Y}
%\item \textbf{W}
%\item \textbf{Z}
\end{list}
