\label{analisi_soluzioni}
\subsection{Gestione del tempo}
\label{tempo}
La competizione dovrebbe essere regolata e scandita da un orologio. La funzione
di tale orologio è quella di permettere
di tenere traccia della durata degli eventi. \`{E} necessario ad esempio ai
(thread) concorrenti per decidere quanto attendere all'entrata
di un tratto prima che esso si liberi.\\
L'orologio potrebbe essere di due tipi:
\begin{description}
\item[Assoluto]:\\
ovvero un orologio che fa riferimento ad un tempo assoluto. L'orologio dispone
di un suo flusso temporale costante 
interno a cui le entità esterne attingono.
\item[Relativo]:\\
in questo caso non esiste un unico orologio esterno di riferimento. Ogni entità
possiede un orologio interno con il proprio istante
zero che viene fatto procedere incrementalmente dall'entità stessa.
\end{description}
La prima soluzione non si presta ad un sistema come quello in analisi per i
seguenti motivi:
\begin{itemize}
\item un orologio assoluto non tiene in considerazione i ritardi dati dalle
caratteristiche del sistema su cui il simulatore dovrebbe
essere eseguito. Ad esempio non tiene conto del tempo che intercorre da quando
un thread è sulla coda dei pronti a quando gli viene assegnata la CPU.
Di conseguenza un concorrente (che supponiamo essere rappresentato da un thread)
che si annunci pronto ad attraversare un tratto all'istante
\emph{t}, potrebbe (con alta probabilità) iniziare effettivamente ad
attraversarlo ad un istante \emph{t+$\epsilon$}. In questo caso
il risultato della simulazione sarebbe errato. Il tempo di attraversamento di
tale concorrente, cioè, sarebbe slittato di $\epsilon$.
\item il flusso di un orologio assoluto è unico. Ovvero, l'orologio assoluto
segue un'unica linea temporale. La natura concorrente del sistema
invece vorrebbe che l'orologio segua una linea per ogni thread in esecuzione.
Supponendo per esempio che ogni concorrente sia un thread,
i concorrenti otterrebbero quanti di esecuzione in modo sequenziale anche se
nella realtà simulata starebbero teoricamente svolgendo
attività in parallelo. Nel caso quindi due o più thread debbano chiedere
l'istante di tempo contemporaneamente, 
l'orologio verrebbe interrogato in modo sequenziale fornendo due tempi diversi.
\end{itemize}
I problemi sopra citati sarebbero invece superati implementando tempo in modo
relativo. Durante lo svolgimento della competizione, 
determinati eventi richiedono un preciso
intervallo di tempo per essere portati a termine (come l'attraversamento di un
tratto da parte di un concorrente). 
Questi eventi influenzano altri eventi (ad esempio
l'attraversamento dello stesso tratto da un altro concorrente). 
% Influenzano inoltre l'informazione che un osservatore esterno riceve (lo
% spettatore vede che un concorrente ha impiegato un tempo \emph{t} per
% attraversare un tratto di pista).
La gestione del tempo in modo relativo vuole, in questo caso, che l'istante in
cui un tratto si libera sia dato
non dall'istante reale in cui l'ultimo concorrente ha lasciato il tratto (o
meglio, in cui l'ultimo thread ha rilasciato
la risorsa tratto), ma dall'accumulo degli intervalli temporali richiesti da
tutti i concorrenti per attraversarlo (assumendo che il tratto
sia a molteplicità uno).\\
Da questo esempio si evince che la competizione non possa essere regolata da un
unico flusso temporale assoluto. Ogni risorsa il cui uso
coinvolga fattori temporali (una traiettoria ad esempio), deve
possedere un ``orologio'' interno che venga incrementato di un dato offset ogni
volta un evento avvenuto su tale risorsa influenzi i tempi
prodotti dai suoi futuri utilizzatori. L'orologio interno chiaramente ha un
istante zero uguale a tutti gli orologi presenti nel sistema
e non ha vita propria. Cambia cioè solo se qualcuno lo aggiorna.\\
\subsection{Sorpassi impossibili}
\label{sorpassi_impossibili_soluzioni}
Il problema dei sorpassi impossibili è dovuto essenzialmente a due fattori:
\begin{itemize}
\item natura concorrente del sistema
\item risorse condivise (tratti) fra thread concorrenti
\end{itemize}
Esisterebbe un metodo semplice e veloce di risolvere entrambi i problemi
elencati: ridurre tutti i concorrenti ad un unico thread.
Dovrebbe cioè essere presente un thread destinato a calcolare lo svolgimento
della gara, utilizzando come  parametri decisionali
le caratteristiche dei concorrenti e i dettagli del circuito. In questo modo
sparirebbe il problema delle risorse condivise fra
i thread designati a svolgere il ruolo di concorrenti di gara. Tuttavia sarebbe
una scelta poco elegante che porterebbe
ad annullare più che risolvere le problematiche di concorrenza.\\
Una soluzione più valida dovrebbe invece mantenere una reale concorrenza fra
concorrenti, rispecchiando quanto accade nella realtà.
Bisognerebbe quindi istanziare un task per ogni concorrente. Per simulare una
gara di formula 1 adeguatamente sarebbe poi necessario
fornire una serie di risorse che rappresentino il circuito. Emerge qui il
concetto di \textbf{tratto}: un singolo segmento di pista.
Per rendere la simulazione più realistica, un tratto dovrebbe avere molteplicità
limitata per permettere ad un numero finito di macchine
di attraversarlo contemporaneamente. Per ora quindi le entità usate sarebbero:
\begin{itemize}
\item \textbf{Concorrente}, entità attiva istanziata una volta per ogni
concorrente;
\item \textbf{Tratto}, entità passiva condivisa fra concorrenti e quindi ad
accesso mutuamente esclusivo (con molteplicità arbitraria non infinita);
\end{itemize}
In questa soluzione ogni concorrente procede verso il tratto N solo dopo aver
passato il tratto N-1. Per ottenere il tratto N deve attendere
che la risorsa si liberi. Questa bozza di soluzione però espone un problema
fondamentale: non essendoci accumulo di risorse,
un concorrente che abbia ottenuto un tratto N e debba richiedere
l'attraversamento del tratto N+1, dovrà rilasciare N e poi ottenere N+1.
Nell'intervallo
di tempo in mezzo ai due eventi, il thread che gestisce il concorrente potrebbe
essere prerilasciato a far passare avanti altri concorrenti
in modo incontrollato. In questo caso si potrebbe facilmente presentare lo
scenario esposto durante l'enunciazione del problema nella sezione
precedente.\\
Permettendo l'accumulo di risorse da parte dei thread si presenterebbe la
possibilità di stallo. Con un numero di concorrenti pari al numero di
segmenti, se ogni concorrente per procedere avesse bisogno del tratto corrente e
di quello successivo e se ogni concorrente
fosse su un tratto diverso, lo stallo non esiterebbe a presentarsi.
Il problema dello stallo verrà comunque trattato più dettagliatamente qualche
sezione più avanti.\\
Per gestire quindi il problema è necessario introdurre delle strutture che
permettano di creare una certa dipendenza fra eventi. Più precisamente,
bisogna garantire che ogni concorrente sappia quale sia il momento adatto per
procedere alla richiesta del tratto successivo senza il rischio
di teletrasportarsi davanti ad un altro concorrente o di creare incoerenza
temporale. Allo stesso modo, il thread che rappresenta il concorrente
deve poter essere prerilasciato senza che ciò implichi un potenziale problema.
Come vedremo meglio nella sezione riguardante l'esplicazione
della soluzione, il risultato desiderato viene ottenuto regolando l'accesso ai
tratti tramite code. Questo, insieme agli orologi relativi,
permetteranno ad un concorrente di sapere quando la sua esecuzione non potrà
creare conflitti con altri concorrenti.
\subsection{Determinismo} 
Il determinismo della simulazione non può essere ottenuto tramite una precisa
idea o soluzione. Il suggerimento, come già enunciato,
dovrebbe semplicemente essere che l'architettura e il funzionamento del sistema
siano indipendenti da come il sistema operativo
sottostante sia stato progettato. Si può già dire che implementando un orologio
relativo, parte del non determinismo regalato dallo scheduler
venga eliminato. L'avanzamento il funzionamento dell'orologio relativo dipende
esclusivamente dalle scelte progettuali effettuate per 
il simulatore di formula 1.\\
Allo stesso modo, l'implementazione di un protocollo di attraversamento dei
tratti che segua le direttive suggerite nella sottosezione precedente
dovrebbe evitare che la gestione dei processi dello scheduler possa introdurre
non determinismo indesiderato.\\
Rimane ancora il problema del comportamento non prevedibile della rete. Per
quanto riguarda i ritardi di rete, devono essere trattati come
se tali ritardi fossero quelli introdotti dallo scheduler. Ovvero: il sistema
non deve dipendere dalla puntualità delle chiamate remote.\\
Per quanto riguarda invece i fault di connessione (ad esempio un nodo per
qualche motivo smette di essere connesso al sistema), il problema 
si potrebbe risolvere applicando ridondanza alle componenti remote, ovvero
dislocando le stesse componenti su più nodi in modo da attivarne
di alternative in caso di fault. Ma si è pensato che l'introduzione di questa
caratteristica nel sistema avrebbe richiesto uno sforzo progettuale
maggiore rispetto alle pianificazioni. Inoltre, questo tipo di problema potrebbe
essere facilmente associato ad un problema esistente anche nella
realtà per determinate scelte delle componenti distribuite. Come si vedrà in
seguito, infatti, si è scelto di distribuire i box e le TV per
la visione della gara. \`{E} plausibile che nella realtà una TV venga spenta
(chiudendo il contatto con il sistema) o che un box perda
il contatto radio con il proprio concorrente. In questi casi i fault vanno
gestiti in modo che non risultino in errori durante la competizione.
\subsection{Componenti di non determinismo }
\label{componenti_non_determinismo}
Come introdotto nel paragrafo \ref{non_determinismo} le componenti di non
determinismo possono essere desiderabili se opportunamente gestite, o non
desiderabili, se non portano valore aggiunto al prodotto. Nella realizzazione di
un simulatore di formula 1 la scelta di introdurre delle componenti che possono
modificare l'andamento della gara in maniera non prevedibile prima
dell'esecuzione aiuta a simulare in maniera migliore l'andamento di una vera
gara di automobili. Le soluzioni possibili che sono state considerate per
realizzare il progetto \emph{F1\_Sim} sono principalmente tre:
\begin{itemize}
\item Nessuna possibilit\`{a} di componenti di non determinismo
\item Possibilit\`{a} di componenti di non determinismo opportunamente gestite
\item Possibilit\`{a} di componenti che simulino il non determinismo
opportunamente gestite
\end{itemize}
La prima opzione prevede la mancanza di interazione dell'utente che sta
visualizzando la simulazione oltre che la mancanza di parti non deterministiche,
cioè zone del progetto dove non si ha la padronanza e non si prevede il
comportamento. Ad esempio affidarsi al metodo di gestione dei processi dello scheduler per far funzionare la
simulazione porta inevitabilmente a problemi di non predicibilità. La
seconda opzione permette zone del progetto non controllabili e verificabili
mentre la terza opzione permette delle componenti che rendono l'esecuzione
più reale e non verificabile a priori ma comunque gestita e controllata
già a livello di progettazione. Un esempio di componente è quella che
permette l'interazione di un utente che visualizza l'esecuzione del simulatore.
Tale azione pu\`{o} provocare, ad esempio, il rientro ai box del concorrente e
questo andr\`{a} certamente a influire in modo non prevedibile a priori
sull'andamento della gara.\\
Una buona soluzione che pu\`{o} essere adottata è data quindi dalla presenza di
possibilit\`{a} di interazione dell'utente esterno con l'ambiente di simulazione
ad esempio permettendo di provocare un pitstop non previsto e di impedire
(tramite una buona progettazione) la presenza di componenti di che portano non
determinismo al sistema, come ad esempio utilizzare le istruzioni di sleep per
creare l'ordine di uscita da un tratto (cioè accodare i concorrenti man mano che
vengono risvegliati).
\subsection{Stalli }
Per quanto riguarda gli stalli si \`{e} visto nel paragrafo \ref{stalli} quali
siano le condizioni perch\`{e} si verifichino. Le soluzioni da adottare sono,
semplicemente, l'evitare che si presentino le 4 pre-condizioni simultaneamente
non andando incontro, cos\`{i}, al verificarsi di uno stallo.
Trattando nel progetto entit\`{a} attive e passive si possono creare situazioni
che, se non controllate, portano inevitabilmente al verificarsi di situazioni di stallo. 
Le entità attive in un sistema distribuito sono le parti che solitamente utilizzano le 
entità passive. Uno degli esempi \`{e} gi\`{a} descritto nel paragrafo
\ref{sorpassi_impossibili_soluzioni}. e verr\`{a} qui approfondito. Se
fornissimo la possibilit\`{a} a un concorrente  di accumulare risorse per
evitare il problema del teletrasporto (segnalato nel paragrafo
\ref{sorpassi_impossibili}) andremo incontro certamente a una situazione di
stallo nonch\`{e} alla realizzazione di un sistema che rischia di perdere la
concorrenza in quanto, avendo tutte le risorse, il concorrente che pu\`{o}
eseguire sulla pista (insieme di tratti, definiti precedentemente) \`{e} solo
uno. Nel caso non riesca ad ottenere tutta la pista (ad esempio perch\`{e} un
altro concorrente \`{e} stato prerilasciato prima di liberare tutte le risorse)
la simulazione rimarrebbe ferma in una situazione di stallo. La soluzione \`{e}
quindi di studiare e progettare un sistema che preveda la mancanza assoluta di
comportamenti anomali (come sorpassi impossibili), dia la possibilit\`{a} di
esecuzione simultanea a pi\`{u} concorrenti e non presenti per questo situazioni
di stallo. Impedire l'accumulo di tratti da parte di un concorrente ed evitare 
situazioni di attesa circolare sono quindi le basi su cui fondare la soluzione
al problema degli stalli.
\subsection{Realismo fisico}
\label{analisi_realismo_fisico}
Come enunciato nella sezione precedente, il realismo fisico di un simulatore di
formula 1 è ottenuto dall'interazione
concorrente-ambiente statico e concorrente-concorrenti. Verrà ora analizzata la
traccia di soluzione per entrambi:
\begin{description}
\item{Concorrente - ambiente statico:}\\
Per ottenere un minimo livello di realismo da questo punto di vista, è
necessario fornire dei dettagli riguardanti l'aspetto fisico
del circuito. Ogni concorrente, prima di valutare la traiettoria da percorrere
per attraversare il tratto, deve poter conoscere
gli aspetti caratterizzanti di tale traiettoria, quali ad esempio:
\begin{itemize}
\item lunghezza 
\item livello di aderenza
\item angolo
\end{itemize}
Queste caratteristiche, unite a quelle dell'auto (benzina disponibile, tipo di
gomme ed usura, massima accelerazione ecc.)
possono essere utilizzate per dare realismo alla simulazione.
Arricchendo quindi la descrizione del circuito e dei suoi tratti con le
precedenti caratteristiche, il concorrente potrà
valutare per ogni possibile traiettoria la benzina consumata, l'usura delle
gomme, la velocità massima raggiungibile.
\item{Concorrente - concorrenti:}\\
Un livello di interattività fra concorrenti potrebbe essere raggiunto se 
un concorrente potesse controllare, una volta sul tratto, quali concorrenti
siano contemporaneamente presenti sul tratto
per poi valutare quale traiettoria scegliere in base all'``affollamento''.
Esistono chiaramente vari modi di ottenere questo:
\begin{itemize}
\item il concorrente interroga i diversi concorrenti per sapere quale sia la
loro posizione sulla pista. In questo modo potrebbe
avere una visione di insieme della situazione e valutare quindi la scelta della
traiettoria. Per fare ciò, però, sarebbe 
necessario che l'istante \emph{t} del concorrente corrisponda all'istante
\emph{t} dei concorrenti interrogati. In questo caso
è necessario un orologio assoluto che come visto nella sezione \ref{tempo} è
difficilmente realizzabile in un sistema come
quello descritto.\\
Alternativamente ogni concorrente dovrebbe mantenere una storia della sua corsa
fino al momento corrente. Una volta interrogato
sulla sua posizione all'istante \emph{t} dovrebbe ripercorrere all'indietro la
storia fino a collidere con l'istante richiesto
e ritornare l'informazione. Ma una soluzione come questa risulterebbe alquanto
inefficiente.
\item il concorrente dovrebbe poter sapere solo l'informazione utile per
l'attraversamento del tratto corrente. Non è necessario
conoscere la posizione di ogni concorrente o quale concorrente si trova su una
determinata traiettoria. \`{E} sufficiente che
si possa sapere in che istante una data traiettoria si liberi. Questa soluzione
può essere implementata aggiornando un valore
temporale interno alla traiettoria ogni volta che essa venga attraversata,
incrementando (o settando) tale valore in base al 
tempo di attraversamento. Il concorrente potrebbe quindi confrontare l'istante
di tempo relativo
al suo arrivo con quello di liberazione del tratto per poter poi effettuare una
scelta.\\
Questa idea è applicabile ad un sistema come quello dato poiché non necessita di
un orologio assoluto.
\end{itemize}
\end{description}
\subsection{Gestione delle istantanee di gara}
\label{analisi_istantanee}
Un'istantanea di gara è lo stato della gara ad un determinato istante di tempo.
Intuitivamente l'idea più semplice per ottenere
uno snapshot sembrerebbe essere quella di inoltrare la richiesta al sistema
all'istante desiderato, mettere in pausa la gara, 
salvare lo stato e riprendere la competizione dall'istante di pausa. Si
riconsideri però la problematica enunciata nella
sezione \ref{enunciazione_istantanee}:\\
\emph{[...]Bisogna però tenere in considerazione che qualunque entità provveda
allo scatto
dell'istantanea sarà soggetta alle stesse problematiche legate alle altre entità
attive dipendenti dallo scheduler[...]}\\
Il ``mettere in pausa'' la gara, quindi, non è banale. Ogni thread concorrente,
ad esempio, potrebbe essere messo in pausa
in istanti diversi rispetto al \emph{t} richiesto, fornendo quindi uno stato non
valido rispetto \emph{t}. 
Se si considera poi che un orologio assoluto nel sistema si è dimostrato non
plausibile da implementare, l'istante \emph{t} richiesto 
per la pausa non corrisponderebbe ad un'istante
di tempo assoluto di pausa per la gara. All'interno della competizione le
singole entità (che ne necessitano) seguono un orologio interno.\\
\`{E} quindi opportuno che, una volta richiesto lo snapshot di un istante
temporale, lo stato di ogni entità (utile all'istantanea) in 
quell'istante sia disponibile o, alternativamente, l'unità richiedente si metta
in attesa fino alla disponibilità completa dello stato richiesto. 
Bisognerà tenere una storia della gara che permetta di risalire
allo stato globale ad un dato istante.
\subsection{Problematiche di distribuzione}
Per quanto riguarda la robustezza del sistema distribuito le soluzioni possibili
variano in base alla scelta dello standard di comunicazione, della scelta del
tipo di dati che viaggia nella rete e dal livello di distribuzione scelta per il
prodotto finale.
Per quanto riguarda il tipo di middleware di comunicazione la scelta potr\`{a}
cadere tra 
\begin{itemize}
\item CORBA
\item Distributed System Annex 
\item MOM
\end{itemize}
oppure con la costruzione ad hoc di un middleware di comunicazione scritto su
misura per la nostra applicazione.
La scelta dei dati da trasferire fra i vari nodi della rete dipende fortemente
dal tipo di sistema di comunicazione che si sceglie, anche se può comunque
essere orientata verso un utilizzo prevalente delle stringhe o di tipi primitivi
o di una combinazione dei due. Il livello di distribuzione va deciso in base
alle funzionalit\`{a} e alle caratteristiche che si vogliono dare al prodotto in
uscita. Si può pensare di spingere al massimo la distribuzione mettendo ogni
entit\`{a} in nodi separati oppure di ridurre al minimo la distribuzione
mettendo solo degli schermi che visualizzino l'andamento della competizione.
Ovviamente si pu\`{o} anche scegliere una mediazione fra le due soluzioni.Un
esempio \`{e} dato dalla scelta di distribuire componenti che visualizzino
l'andamento della gara (e possano anche interagire, come i box) e lasciare
centralizzata la gara vera e propria, quindi con le entit\`{a} attive e passive
situate nella stessa macchina. Altra scelta fondamentale che va pensata e decisa
in fase di progettazione \`{e} quella del tipo di chiamate da effettuare, se
sincrone o asincrone, considerandone i vantaggi e gli svantaggi. Per quanto
riguarda questo simulatore di formula uno concorrente e distribuito le chiamate
sincrone forniscono uno strumento adatto e semplice per la sincronizzazione fra
oggetti remoti. Inoltre se opportunamente gestite forniscono risultati senza
la presenza di polling o altri meccanismi di recupero dei parametri o dei valori
di ritorno.
\subsection{Avvio del sistema}
Indipendentemente da quali saranno, nel dettaglio dell'implementazione, le
relazioni fra entità e il livello di distribuzione
delle unità componenti il sistema, l'avvio del sistema dovrebbe seguire, per i
due problemi enunciati alla sezione \ref{enunciazione_avvio},
le seguenti linee guida:
\begin{itemize}
\item Assumendo l'esistenza di un nodo centralizzato che ospiterà la maggior
parte del calcolo computazionale legato alla gara e
un numero di nodi distribuiti che interagiranno con esso (anche
bidirezionalmente), le possibilità per la messa in connessione
dei nodi sono 2:
\begin{itemize}
\item Il nodo centrale dispone già di una lista di indirizzi da utilizzare per
contattare i nodi remoti. Ma questo richiederebbe
che nella fase di init o tutti i nodi (come precondizione) siano già avviati, o
che l'unità centrale effettui polling verso tutte
le entità remote fino al loro avvio o, infine, che il nodo centrale si occupi
anche dell'avvio dei nodi remoti. \\
La prima idea richiede un'assunzione troppo forte: con un numero elevato di nodi
remoti potrebbe diventare operazione
lunga sapere quando tutti siano pronti. La seconda comporta uno certo spreco
computazionale: un polling sequenziale dei nodi
potrebbe far attendere per un certo intervallo di tempo un nodo pronto (nel caso
venga avviato subito dopo essere stato interrogato).
Con un insieme di polling paralleli il problema si risolverebbe, ma il sistema
sarebbe comunque poco elastico se inaspettatamente
uno dei nodi previsti dovesse cambiare locazione.
L'ultima idea potrebbe essere una soluzione elegante, ma espone il problema
(presente anche nelle altre due idee di soluzione)
che se le entità remote venissero spostate, la lista dovrebbe essere aggiornata
di volta in volta.\\
Sia chiaro che questo non è un problema non risolvibile in alcun modo, ma si
pensa potrebbe richiedere uno sforzo progettuale maggiore del previsto.
\item Il nodo centrale viene avviato e rimane in attesa che i nodi remoti lo
contattino fornendo le proprie informazioni di posizionamento.
In questo modo i nodi remoti, una volta avviati, dovrebbero conoscere
l'indirizzo del nodo centrale e contattarlo. Così facendo, una volta
che tutti i nodi necessari all'avvio della competizione saranno inizializzati,
il nodo centrale avrà già a disposizione gli indirizzi per la comunicazione
bidirezionale con la certezza che essi siano avviati. I nodi remoti potrebbero
comunque dover effettuare polling in attesa che il nodo centrale sia avviato.
\end{itemize}
\item Per gestire l'avvio delle entità concorrenti, una soluzione plausibile è
quella di organizzare l'inizializzazione del sistema a step.
Fra le entità intercorrono varie relazioni di dipendenza. Alcune necessitano di
dati ottenibili a partire da altre, le quali a loro volta
possono aver bisogno di accedere a risorse disponibili solo ad un determinato
momento della fase di avvio. Per modellare la soluzione
è quindi necessaria una classificazione delle componenti in base alle risorse
richieste. Il primo step dell'avvio dovrà occuparsi di inizializzare
le risorse (attive, passive o reattive che siano) che non richiedano dati
decisi a tempo di esecuzione. Durante gli step successivi
si dovranno man mano avviare le risorse (o parti di risorse) che dipendono dai
dati precedentemente inizializzati. Se un'entità attiva necessita
di dati ottenibili in due o più step diversi per poter considerarsi pronta, sarà
necessario che di step in step venga messa in attesa di tali dati.
Questo può avvenire mettendo l'entità in attesa su una risorsa (finché questa
non raggiunga uno stato adeguato per fornire i dati richiesi)
oppure mettendo l'entità stessa in attesa di messaggi (che potrebbero contenere
i dati richiesti o semplicemente segnalare che è possibile
procedere). Come post-condizione alla fine dell'ultimo step è necessario che
tutte le entità passive e reattive che verranno richieste dopo l'init
siano disponibili e che le risorse attive siano pronte ad iniziare con la reale
esecuzione.
\end{itemize}
\subsection{Stop del sistema}
Si è visto che lo stop del sistema è suddiviso in uno stop riferito alla fine
della gara ed uno legato alla finalizzazione delle risorse allocate.
Di seguito i dettagli:
\begin{itemize}
\item Per lo stop logico si potrebbe delegare alle entità attive che gestiscono
la corsa dei singoli concorrenti l'onere di tenere conto 
dello stato della gara (il numero di giri effettuati ad esempio) e decidere, al
momento opportuno, di procedere con lo stop di quelle che
finiscono la competizione (o per numero di giri effettuati o per impossibilità a
continuare).
\item La finalizzazione delle risorse potrebbe avvenire anche prima che la
competizione finisca se l'utente decide di interrompere anticipatamente
l'esecuzione del simulatore. Di conseguenza, per permettere uno stop definitivo
del sistema, sarà necessario che il processo del simulatore
venga terminato nel momento in cui l'utente chiuda l'applicazione.
\end{itemize}