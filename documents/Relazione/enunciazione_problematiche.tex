Nella seguente sezione verranno esposte le problematiche emerse nel corso dell'analisi del sistema da progettare. Tali
problematiche esistono perchè il sistema presenta le seguenti caratteristiche:
\begin{itemize}
\item il sistema è concorrente. Di conseguenza presenterà un numero maggiore di 2 entità attive
che eseguiranno concorrentemente;
\item alcune risorse sono condivise e accedute quindi concorrentemente;
\item il simulatore viene eseguito su un sistema operativo del cui scheduler non sono conosciute le specifiche;
\item il sistema presenta delle componenti distribuite nella rete;
\end{itemize}
\subsection{Gestione del tempo}
Un simulatore di formula 1 racchiude intrinsecamente dei vincoli in termini di coerenza temporale. 
Una competizione è scandita da istanti di tempo: un istante iniziale, uno finale e vari istanti 
intermedi che segnano, per esempio, la fine della lap di un concorrente, oppure il passaggio di un concorrente da un settore
a quello dopo del circuito. Vi sono inoltre vincoli di coerenza temporale dati dai tempi accumulati nel corso
della gara. Questi potrebbero essere il tempo di attraversamento di un tratto come il tempo necessario
a effettuare un giro. I secondi, chiaramente, dipendono dai primi. 
In un sistema simulato è necessario che per più simulazioni regolate dalle stesse condizioni, 
gli intervalli temporali calcolati siano gli stessi. Questo potrebbe risultare problematico dal momento che
lo scheduler non rispetta vincoli di tempo definiti o comunque conosciuti a priori.
\subsection{Sorpassi impossibili}
Una problematica affrontata nel corso dell'analisi del simulatore da progettare
è stata quella relativa ai sorpassi. Come accennato all'inizio
della sezione, non possono essere fatte assunzioni di determinismo sullo scheduler. 
Ipotizzando un sistema in cui ogni concorrente
corrisponda ad un'entità attiva (task) e in cui ogni tratto del circuito sia una risorsa condivisa fra i task a molteplicità
limitata (come è plausibile pensare per un simulatore di F1), è possibile che si verifichino scenari anomali.\\
Per esempio:
\begin{enumerate}
\item un task concorrente dovrebbe, per questioni temporali, iniziare ad attraversare un tratto e ottenere quindi la risorsa;
\item lo scheduler prerilascia il task e assegna un quanto di tempo ad un altro task concorrente che in termini temporali
è dietro. Il task ottiene la risorsa tratto (la stessa del task precedente);
\item il task ottiene altri quanti e attraversa;
\item lo scheduler prerilascia il task corrente e riassegna il processore al vecchio task il quale anche attraversa il tratto.
\end{enumerate}
Non prestando attenzione a possibilità di questo tipo, se il tratto fosse di molteplicità 1 si avrebbe che un concorrente 
appare alla fine del tratto quando, per questioni fisico/temporali, avrebbe dovuto rimanere dietro.
\subsection{Determinismo}
Come definito in \url{http://it.wikipedia.org/wiki/Simulazione}, \\

\emph{``per simulazione si intende un modello della realtà che consente di valutare e prevedere lo 
svolgersi dinamico di una serie di eventi o processi susseguenti all'imposizione di certe condizioni da parte dell'analista
o dell'utente''}.\\
Il comportamento del sistema deve essere quindi prevedibile. Considerando che invece il comportamento dello 
scheduler sottostante non è prevedibile, si presenta il problema di riuscire a progettare il simulatore in modo
che il comportamento sia del tutto indipendente dall'architettura del sistema operativo su cui viene eseguito. In questo
modo si potrà avere un sistema la cui correttezza sia verificabile e capace di fornire dati consistenti.
\subsection{Componenti di non determinismo (Lorenzo)}
\subsection{Stalli (Lorenzo)}
\subsection{Realismo fisico}
Il realismo fisico di un simulatore di formula uno è determinato principalmente da due fattori:
\begin{itemize}
\item realismo dato dall'interazione tra ogni concorrente e l'ambiente statico. Ovvero, l'impatto che hanno
le caratteristiche fisiche dell'ambiente circostante quali, ad esempio, forma e caratteristiche della pista;
\item realismo dato dall'interazione tra un concorrente e gli altri concorrenti. Questo tipo di realismo 
dipende quindi da caratteristiche dinamiche dell'ambiente e richiede che le valutazioni dei singoli concorrenti
vengano fatte in rapporto allo stato degli altri concorrenti. Viceversa, le scelte dei singoli concorrenti
devono poter influenzare le scelte degli altri concorrenti.
\end{itemize}
\subsection{Gestione delle istantanee di gara}
Requisito essenziale per poter presentare i dati relativi all'andamento della competizione è riuscire ad ottenere una
snapshot della gara in un determinato istanto. Ovvero, dato un istante di tempo \emph{t} o un evento \emph{e}, bisogna
poter risalire allo stato dei concorrenti e della gara in generale in quel momento. In tal modo sarà reso possibile
monitorare l'evolversi della competizione. Bisogna però tenere in considerazione che qualunque entità provvede allo scatto
dell'istantanea sarà soggetta alle stesse problematiche legate alle altre entità attive dipendenti dallo scheduler. 
Di conseguenza è lecito pensare che uno snapshot possa risultare inconsistente se viene eseguito, per motivi di 
prerilascio, metà ad un istante
\emph{t} e l'altra metà ad un istante \emph{t+f}, ad esempio. La distribuzione aggiunge un ulteriore livello di complessità
al problema, poichè il ritardo potrebbe essere causato non solo dallo scheduler, ma anche dalla rete.
\subsection{Robustezza del sistema distribuito (Lorenzo)}
\subsection{Intelligenza artificiale}
Per poter rendere indipendente da controlli esterni l'esecuzione della simulazione, è necessario fornire almeno un accenno
di intelligenza artificiale che riesca a far procedere la gara in modo verosimile. Il problema in questo caso
\subsection{Start del sistema}

\subsection{Stop del sistema}
