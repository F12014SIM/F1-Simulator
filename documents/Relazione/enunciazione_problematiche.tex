Nella seguente sezione verranno esposte le problematiche emerse nel corso dell'analisi del sistema da progettare. Tali
problematiche esistono perchè il sistema presenta le seguenti caratteristiche:
\begin{itemize}
\item il sistema è concorrente. Di conseguenza presenterà un numero maggiore di 2 entità attive
che eseguiranno concorrentemente;
\item alcune risorse sono condivise e accedute quindi concorrentemente;
\item il simulatore viene eseguito su un sistema operativo del cui scheduler non sono conosciute le specifiche;
\item il sistema presenta delle componenti distribuite nella rete;
\end{itemize}
\subsection{Gestione del tempo}
Un simulatore di formula 1 racchiude intrinsecamente dei vincoli in termini di coerenza temporale. 
Una competizione è scandita da istanti di tempo: un istante iniziale, uno finale e vari istanti 
intermedi che segnano, per esempio, la fine della lap di un concorrente, oppure il passaggio di un concorrente da un settore
a quello dopo del circuito. Vi sono inoltre vincoli di coerenza temporale dati dai tempi accumulati nel corso
della gara. Questi potrebbero essere il tempo di attraversamento di un tratto come il tempo necessario
a effettuare un giro. I secondi, chiaramente, dipendono dai primi. 
In un sistema simulato è necessario che per più simulazioni regolate dalle stesse condizioni, 
gli intervalli temporali calcolati siano gli stessi. Questo potrebbe risultare problematico dal momento che
lo scheduler non rispetta vincoli di tempo definiti o comunque conosciuti a priori.
\subsection{Sorpassi impossibili}
\label{sorpassi_impossibili}
Una problematica affrontata nel corso dell'analisi del simulatore da progettare
è stata quella relativa ai sorpassi. Come accennato all'inizio
della sezione, non possono essere fatte assunzioni di determinismo sullo scheduler. 
Ipotizzando un sistema in cui ogni concorrente
corrisponda ad un'entità attiva (task) e in cui ogni tratto del circuito sia una risorsa condivisa fra i task a molteplicità
limitata (come è plausibile pensare per un simulatore di F1), è possibile che si verifichino scenari anomali.\\
Per esempio:
\begin{enumerate}
\item un task concorrente dovrebbe, per questioni temporali, iniziare ad attraversare un tratto e ottenere quindi la risorsa;
\item lo scheduler prerilascia il task e assegna un quanto di tempo ad un altro task concorrente che in termini temporali
è dietro. Il task ottiene la risorsa tratto (la stessa del task precedente);
\item il task ottiene altri quanti e attraversa;
\item lo scheduler prerilascia il task corrente e riassegna il processore al vecchio task il quale anche attraversa il tratto.
\end{enumerate}
Non prestando attenzione a possibilità di questo tipo, se il tratto fosse di molteplicità 1 si avrebbe che un concorrente 
appare alla fine del tratto quando, per questioni fisico/temporali, avrebbe dovuto rimanere dietro.
\subsection{Determinismo}
Come definito in \emph{Simulation: The Engine Behind The Virtual World, Roger D. Smith, Chief Scientist, ModelBenders LLC}:

\emph{``Simulation is the process of designing a model of a real or imagined system and
conducting experiments with that model. The purpose of simulation experiments is to
understand the behavior of the system or evaluate strategies for the operation of the system.''}

%\emph{``per simulazione si intende un modello della realtà che consente di valutare e prevedere lo 
%svolgersi dinamico di una serie di eventi o processi susseguenti all'imposizione di certe condizioni da parte dell'analista
%o dell'utente''}.\\
Considerando quindi che la simulazione deve permettere di comprendere il comportamento del sistema, è necessario che il sistema abbia 
un comportamento prevedibile. Ciò non significa che ogni componente di non determinismo sia non desiderata. Il non determinismo
può essere inserito in modo controllato, ovvero consapevoli del contensto e dei momenti in cui esso si possa verificare. Soprattutto,
il nondeterminismo deve rispecchiare un eventuale non determinismo presente anche nel sistema reale e non solo in quello simulato.\\
Il comportamento dello scheduler sottostante il sistema che verrà sviluppato non è prevedibile. Considerando che il non determinismo
introdotto dallo scheduler non è controllabile, si presenta
il problema di riuscire a progettare il simulatore in modo
che il comportamento sia del tutto indipendente dall'architettura del sistema operativo su cui viene eseguito. In questo
modo si potrà avere un sistema la cui correttezza sia verificabile e capace di fornire dati consistenti.
\subsection{Componenti di non determinismo (Lorenzo)}
\label{non_determinismo}
Dopo aver analizzato le problematiche dovute al determinismo del progetto si possono pensare anche ai fattori che diano del non determinismo.
Per non determinismo si intende la possibilit\`{a} di non prevedere precisamente l'andamento della gara a priori. Questa componente pu\`{o} essere non desiderabile per certi aspetti mentre, se gestita, pu\`{o} dare del valore aggiunto alla simulazione. Nel caso di \emph{F1\_Sim} \`{e} desiderabile riuscire a gestire il non determinismo, che da quindi valore aggiunto al progetto.
Esempi di non determinismo che si possono trovare progettando un simulatore e che possono portare valore aggiunto sono ad esempio interazioni dell'utente esterno al sistema tramite azioni esplicite .
\subsection{Stalli (Lorenzo)}
\label{stalli}
In un sistema concorrente lo stallo è una delle probelmatiche pi\`{u} importanti da affrontare. Per stallo si intende lo stato in cui nessun processo pu\`{o} pi\`{u} eseguire. Affinch\`{e} si verifichi lo stallo di devono verificare tutte le 4 pre-condizioni ben riconoscibili:
\begin{itemize}
\item {Mutua Esclusione :} assicura che, a ogni instante, non pi\`{u} di un processo abbia possesso di una risorsa (fisica o logica) condivisa.La sequenza di azioni che opera sulla risorsa \`{e} detta sezione critica. Nel realizzare un simulatore di Formula 1 bisogner\`{a} fare in modo che prevedendo l'uso esclusivo di risorse, come ad esempio pezzi di tracciato, queste non pregiudichino il buon funzionamento del sistema.
\item{Cumulazione di risorse (hold-and-wait) :} i processi possono accumulare risorse e trattenerle mentre attendono di acquisirne altre
\item{Assenza di prerilascio :} le risorse vengono rilasciate solo volontariamente
\item{Attesa Circolare :} un processo attende almeno una risorsa in possesso del successivo processo in catena. Nella progettazione di un sistema distribuito bisogner\`{a} quindi evitare la formazione di catene chiuse di processi che attendono una risorsa.
\end{itemize}
Nella progettazione e realizzazione del sistema si dovr\`{a} quindi fare attenzione per evitare il verificarsi delle quattro condizioni, impedendo cos\`{i} gli stalli. Le possibili problematiche che portano alla presenza di uno stallo in un simulatore automobilistico sono principalmente la possibilità di accumulo delle risorse (ad esempio possesso della pista per gareggiare), l'esecuzione garantita, in certe zone, solo a un concorrente alla volta (in un tratto di strada nello spazio occupato da un autovettura non è possibile che ce ne siano altre nello stesso istante di tempo), la volontarietà del rilascio di una risorsa (un concorrente che non ha finito di attraversare un pezzo di pista non può lasciarla percorrere ad altri).
\subsection{Realismo fisico}
Il realismo fisico di un simulatore di formula uno è determinato principalmente da due fattori:
\begin{itemize}
\item realismo dato dall'interazione tra ogni concorrente e l'ambiente statico. Ovvero, l'impatto che hanno
le caratteristiche fisiche dell'ambiente circostante quali, ad esempio, forma e caratteristiche della pista;
\item realismo dato dall'interazione tra un concorrente e gli altri concorrenti. Questo tipo di realismo 
dipende quindi da caratteristiche dinamiche dell'ambiente e richiede che le valutazioni dei singoli concorrenti
vengano fatte in rapporto allo stato degli altri concorrenti. Viceversa, le scelte dei singoli concorrenti
devono poter influenzare le scelte degli altri concorrenti.
\end{itemize}
\subsection{Gestione delle istantanee di gara}
Requisito essenziale per poter presentare i dati relativi all'andamento della competizione è riuscire ad ottenere una
snapshot della gara in un determinato istanto. Ovvero, dato un istante di tempo \emph{t} o un evento \emph{e}, bisogna
poter risalire allo stato dei concorrenti e della gara in generale in quel momento. In tal modo sarà reso possibile
monitorare l'evolversi della competizione. Bisogna però tenere in considerazione che qualunque entità provvede allo scatto
dell'istantanea sarà soggetta alle stesse problematiche legate alle altre entità attive dipendenti dallo scheduler. 
Di conseguenza è lecito pensare che uno snapshot possa risultare inconsistente se viene eseguito, per motivi di 
prerilascio, metà ad un istante
\emph{t} e l'altra metà ad un istante \emph{t+f}, ad esempio. La distribuzione aggiunge un ulteriore livello di complessità
al problema, poichè il ritardo potrebbe essere causato non solo dallo scheduler, ma anche dalla rete.
\subsection{Robustezza del sistema distribuito (Lorenzo)}
Progettando un sistema distribuito bisogna tenere in considerazione alcune problematiche che, se ben risolte, portano a un prodotto distribuito robusto.
I punti principali su cui focalizzare l'attenzione sono
\begin{itemize}
\item Apparenza all'utente di un sistema unitario e non l'insieme di pi\`{u} elaboratori
\item Comunicazione fra elaboratori nascosta all'utente
\item Scelta del livello di distribuzione
\item Architettura invariata rispetto al sistema in locale
\end{itemize}
Simulando una competizione di formula uno bisognerà fare in modo che la comunicazione (ad esempio concorrente - box) sia sicura e robusta, in quanto, come nella realtà, può precludere il buon andamento della gara. Inoltre bisogna riuscire a fornire funzionalità adatte ad eventuali osservatori esterni (che non influiscono sulla gara) per coinvolgerli in prima persona con la gara, un po' come avviene nella realtà con le tv collegate direttamente dall'autodromo ma visibili in tutto il mondo. Tutto questo porta a possibili problemi dovuti alla presenza della rete; tali problemi vanno gestiti e considerati per dare robustezza al sistema in uscita, dando l'idea all'utente di avere la simulazione in locale.
\subsection{Intelligenza artificiale}
Per poter rendere indipendente da controlli esterni l'esecuzione della simulazione, è necessario fornire almeno un accenno
di intelligenza artificiale che riesca a far procedere la gara in modo verosimile. Il problema in questo caso
\subsection{Avvio del sistema}
\label{enunciazione_avvio}
L'avvio del sistema presenta dei problemi legati alla natura concorrente e distribuita dello stesso.\\
In dettaglio:
\begin{itemize}
\item la natura distribuita introduce il problema della messa in connessione dei nodi. Quando un nodo si connette
è necessario che esso sappia dove sono dislocati gli altri nodi (o almeno quelli necessari) e che gli altri nodi
possano reperire quello appena connesso.
\item la natura concorrente invece introduce dei problemi relativi alla comunicazione delle entità attive. Tali entità
saranno presumibilmente messe in comunicazione o tramite risorse condivise o in modo diretto. Si prospetta quindi la
possibilità che in fase di avvio (essendo l'avvio un processo sequenziale) alcune entità richiedano la connessione con 
altre che non siano ancora pronte o allocate, causando un fault in alcuni casi, in altri un'esecuzione con risultati 
errati.
\end{itemize}
\subsection{Stop del sistema}
Nel caso specifico di un simulatore di formula 1, lo stop deve avvenire innanzitutto a livello logico. Ovvero, deve essere
possibile al sistema poter capire quando la gara è finita in modo da poterlo annunciare.\\
Una volta che la competizione risulti essere completata, le risorse non più necessarie devono essere deallocate e i task
fermati.\\
Quando il sistema viene interrotto definitivamente, non devono più essere presenti server in attesa di connessioni o
thread attivi.