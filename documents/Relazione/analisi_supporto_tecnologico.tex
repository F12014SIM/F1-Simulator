\subsection{Scelta dei linguaggi}
Per il progetto sono stati utilizzati principalmente due linguaggi (+1 in piccola parte):
\begin{itemize}
\item \textbf{ADA}: il linguaggio \`{e} stato utilizzato per sviluppare i layer sottostanti allo strato di presentazione. Si \`{e} valutato infatti,
dopo le tematiche affrontate durante il corso, che sarebbe stato pi\`{u} agevole sviluppare un sistema multi-threaded e caratterizzato
da forti vincoli di affidabilit\`{a} e determinismo utilizzando tale linguaggio. Il forte sistema di tipi e le primitive a supporto della gestione
dei thread e delle risorse condivise si sono infatti rivelati decisivi avere un maggior controllo sul sistema in fase di sviluppo gi\`{a} 
a compile-time. Pi\`{u} precisamente, il linguaggio stesso ci avrebbe (e ci ha) vincolato a scrivere del codice logicamente corretto senza
troppe iterazioni di debug.
\item \textbf{Java}: per lo sviluppo delle interfacce grafiche \`{e} stato invece utilizzato Java. La scelta \`{e} stata dettata da vari motivi.
Primo fra tutti la competenza. \`{E} sembrato pi\`{u} intuitivo poter gestire un argomento come l'interfaccia utente usando tale linguaggio.
Non risultando infatti critiche a livello di logica e occupando il 15\% dell'intero progetto, si \`{e} pensato che per le interfacce sarebbe
stato pi\`{u} che sufficiente. Anche per la competenza gi\`{a} acquisita di librerie come \emph{awt} e \emph{swing}.\\
Seconda motivazione, la portabilit\`{a}. La parte logica del sistema \`{e} stata concepita per girare su una macchina con il supporto di un numero di
nodi massimo pari alla quantit\`{a} di concorrenti. La TV per la visualizzazione della gara potrebbe invece essere potenzialmente avviata
su 1000 macchine diverse. \`{E} parsa quindi una scelta abbastanza furba offrire la possibilit\`{a} di utilizzare tale interfaccia in qualunque
computer appoggiandosi alla JVM invece che doverla ricompilare e magari riadattare per ogni macchina.
\item \textbf{Bash}: questo linguaggio di scripting \`{e} stato usato per la configurazione e l'avvio del sistema.
\end{itemize}
\subsection{Distribuzione e interoperabilit\`{a} fra linguaggi}

