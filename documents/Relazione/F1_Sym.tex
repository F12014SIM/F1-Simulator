\documentclass[a4paper]{article}
\let\subsubsubsection\paragraph
\let\subsubsubsubsection\subparagraph
\setcounter{secnumdepth}{4}
\setcounter{tocdepth}{4}
\usepackage[latin1]{inputenc}
\usepackage[italian]{babel}
\usepackage{marvosym}
\usepackage{graphicx}
\usepackage{lscape}
\usepackage{fancyhdr}
\usepackage{totpages}
\usepackage{enumerate}
\usepackage{float}
\usepackage{color}
\usepackage[pdftex]{hyperref}
\usepackage{listings}
\usepackage{tabularx}
\hypersetup{colorlinks,breaklinks,linkcolor=blue,urlcolor=black}

\renewcommand{\baselinestretch}{1.25}

\pagestyle{fancy} 

\makeindex

% Definizione di nuovi "tokens" (e valori che possono assumere)
\newtoks\titolo
\newtoks\sottotitolo
\newtoks\filename
\newtoks\data
\newtoks\versione
\newtoks\distribuzione

% Titolo documento + titolo a pie' pagina e data (tokens)
\titolo={F1 Symulator} 
\sottotitolo={Progetto di Sistemi concorrenti e distribuiti} 
\data={02 Luglio 2009}

% Informazioni documento (tokens)
\filename={Telemedicina.pdf}
\versione={0.3}
\distribuzione={Prof. Vardanega Tullio \\
		     	& Miotto Nicola \\
			& Nesello Lorenzo
}

% Header (left, center, right)
\lhead{} 
\chead{}
\rhead{}
\renewcommand{\headrulewidth}{0.4pt}
% Footer (left, center, right)
\lfoot{} \cfoot{} \rfoot{\thepage/\pageref{TotPages}}
\renewcommand{\footrulewidth}{0.4pt}

% Inizio documento LaTeX
\begin{document} %produce il titolo a partire dai comandi \title, \author e \date

\begin{center}
\vspace*{1,0 cm}
\huge\textbf{\the\titolo} \\ %LARGE != large
\vspace{0,2 cm}
\large\the\sottotitolo \\
\vspace{0,4 cm}
\large\the\data
\end{center}
\begin{center}
\vspace{1,75 cm}

% Sommario
\begin{abstract} 
\begin{center}
Descrizione del progetto di telemedicina per il corso di Sviluppo e Gestione di Progetti.
\end{center}
\end{abstract}
\vspace{1,50 cm}

% Informazioni documento
\textbf{Informazioni documento} \\ \vspace{0.5cm}
\begin{tabular}{r | l }
\textbf{Nome file}      & \the\filename         \\
\textbf{Versione}       & \the\versione         \\
\textbf{Distribuzione}  & \the\distribuzione    \\ \\
\end{tabular}
\vspace{0,3cm}
\end{center}

\newpage

\tableofcontents

\section{Competizione}
\section{Circuito}
\section{Pista}
\section{Box}
\section{Tratto}
\section{Traiettoria}
\section{Concorrente}
\section{Sistema di controllo}
\section{Interfaccia di monitoraggio}

\end{document}