\documentclass[a4paper]{book}
\let\subsubsubsection\paragraph
\let\subsubsubsubsection\subparagraph
\setcounter{secnumdepth}{4}
\setcounter{tocdepth}{4}
\usepackage[latin1]{inputenc}
\usepackage[italian]{babel}
\usepackage{marvosym}
\usepackage{graphicx}
\usepackage{lscape}
\usepackage{fancyhdr}
\usepackage{totpages}
\usepackage{enumerate}
\usepackage{float}
\usepackage{color}
\usepackage[pdftex]{hyperref}
\usepackage{listings}
\usepackage{tabularx}
\hypersetup{colorlinks,breaklinks,linkcolor=blue,urlcolor=black}

\usepackage{fancyhdr}
\newcommand{\fncyblank}{\fancyhf{}}
\newenvironment{abstract}%
{\fncyblank\null\vfill\begin{center}%
\bfseries\abstractname\end{center}}%
{\vfill\null}

\renewcommand{\baselinestretch}{1.25}

\pagestyle{fancy} 

\makeindex

% Definizione di nuovi "tokens" (e valori che possono assumere)
\newtoks\titolo
\newtoks\sottotitolo
\newtoks\filename
\newtoks\data
\newtoks\versione
\newtoks\distribuzione

% Titolo documento + titolo a pie' pagina e data (tokens)
\titolo={F1 Symulator} 
\sottotitolo={Progetto di Sistemi concorrenti e distribuiti} 
\data={02 Luglio 2009}

% Informazioni documento (tokens)
\filename={Telemedicina.pdf}
\versione={0.3}
\distribuzione={Prof. Vardanega Tullio \\
		     	& Miotto Nicola \\
			& Nesello Lorenzo
}

% Header (left, center, right)
\lhead{} 
\chead{}
\rhead{}
\renewcommand{\headrulewidth}{0.4pt}
% Footer (left, center, right)
\lfoot{} \cfoot{} \rfoot{\thepage/\pageref{TotPages}}
\renewcommand{\footrulewidth}{0.4pt}

% Inizio documento LaTeX
\begin{document} %produce il titolo a partire dai comandi \title, \author e \date

\begin{center}
\vspace*{1,0 cm}
\huge\textbf{\the\titolo} \\ %LARGE != large
\vspace{0,2 cm}
\large\the\sottotitolo \\
\vspace{0,4 cm}
\large\the\data
\end{center}
\begin{center}
\vspace{1,75 cm}

% Sommario
\begin{abstract} 
\begin{center}
Realzione sul progetto di Sistemi Concorrenti e Distribuiti.
\end{center}
\end{abstract}
\vspace{1,50 cm}

% Informazioni documento
\textbf{Informazioni documento} \\ \vspace{0.5cm}
\begin{tabular}{r | l }
\textbf{Nome file}      & \the\filename         \\
\textbf{Versione}       & \the\versione         \\
\textbf{Distribuzione}  & \the\distribuzione    \\ \\
\end{tabular}
\vspace{0,3cm}
\end{center}

\newpage

\tableofcontents

\chapter{Descrizione del progetto}
\section{Progetto}
Il progetto riguarda l'analisi e la risoluzione delle problematiche di progettazione di un simulatore concorrente e distribuito di una competizione sportiva assimilabile a quelle automobilistiche di Formula 1.
Il sistema da simulare dovr� prevedere:
\begin{itemize}
    \item{un circuito selezionabile in fase di configurazione, dotato della pista e della corsia di rifornimento, ciascuna della quali soggette a regole congruenti di accesso, condivisione, tempo di percorrenza, condizioni atmosferiche, ecc.}
    \item{un insieme configurabile di concorrenti, ciascuno con caratteristiche specifiche di prestazione, risorse, strategia di gara, ecc.}
    \item{un sistema di controllo capace di riportare costantemente, consistentemente e separatamente, lo stato della competizione, le migliori prestazioni (sul giro, per sezione di circuito) e anche la situazione di ciascun concorrente rispetto a specifici parametri tecnici}
    \item{una particolare competizione, con specifica configurabile della durata e controllo di terminazione dei concorrenti a fine gara.}
\end{itemize}

\chapter{Problematiche}
\section{Introduzione}
In questo capitolo verranno analizzate le problematiche legate alla realizzazione di un sistema concorrente e distribuito.
\section{Problematiche di concorrenza}
\label{problematiche_concorrenza}
% - Predicibilità della gara
% - Corretto svolgimento della gara rispetto agli input forniti, in modo indipendente dalle funzionamento
% delle componenti a basso livello de sistema operativo (scheduler....)
% - Evitare race condition su risorse condivise
% - Evitare stalli del sistema
% - Assicurarsi che vi sia almeno un istante t in cui due o più task eseguano in concorrenza
% - Coerenza temporale
% - Evitare starvation
Nel corso dell'analisi ad alto livello del progetto, sono emerse numerose problematiche legate alla coesistenza nel 
sistema di pi\`{u} \emph{task} concorrenti e di risorse tra di essi condivise. \\
\subsubsection{Percorrenza concorrente della pista}
Come da specifica, il sistema dovr\`{a} prevedere una pista per lo svolgimento della gara. I partecipanti percorreranno quindi i tratti del circuito in modo concorrente e questo pone gi\`{a} numerose problematiche per il corretto svolgimento della competizione.
Si pu\`{o} vedere ogni tratto come una risorsa condivisa fra i concorrenti della gara. Ciascuna risorsa avr\`{a} una molteplicit\`{a} limitata, coerentemente a quanto accade nella realt\`{a}. Questo porta inevitabilmente a due problematiche da affrontare:
\begin{enumerate}
\item I concorrenti dovranno attraversare ogni tratto in modo da non violare i limiti di molteplicit\`{a} imposti. Bisogner\`{a} quindi impedire che ad un istante \textbf{t} dello svolgimento della gara, su un tratto con molteplicit\`{a} \textbf{n} vi siano, contemporaneamente, \textbf{n+1} auto che lo stanno attraversando;
\item Evitare l'effetto ``teletrasporto''. Uno possibile scenario che si potrebbe infatti presentare nel corso della gara potrebbe essere il seguente (in caso di progettazione poco attenta):
Un tratto $Tr_N$ a molteplicit\`{a} \textbf{1} viene attraversato da \textbf{2} auto, \textbf{A} e \textbf{B}. Logicamente, quindi, se \textbf{A} arriva prima di \textbf{B}, all'inizio del tratto $Tr_{N+1}$ l'ordine dovr\`{a} essere mantenuto. Ad alto livello si potrebbe pensare, come soluzione plausibile, di affidare ad \textbf{A} la risorsa $Tr_N$ e, una volta effettuato virtualmente l'attraversamento, farla rilasciare e affidarle il tratto $Tr_{N+1}$, mentre \textbf{B} star\`{a} cercando di ottenere$Tr_N$ e percorrerlo.
Questa soluzione non tiene per\`{o} in considerazione le problematiche legate alla gestione dei processi dallo scheduler. Potrebbe infatti accadere che:
\begin{itemize}
\item \textbf{A} ottiene $Tr_N$ e \textbf{B} rimane in attesa;
\item \textbf{A} rilascia $Tr_N$ e viene prerilasciato dallo scheduler;
\item \textbf{B} ottiene $Tr_N$, lo rilascia e successivamente ottiene $Tr_{N+1}$ prima di essere prerilasciato dallo scheduler;
\item \textbf{A} \`{e} di nuovo attivo ma si trova in una posizione non coerente con quanto previsto: \`{e} avvenuto un sorpasso in una zona non consentita.
\end{itemize}
Da questo emerge il problema della gestione dei processi a livello di sistema operativo. Non \`{e} possibile infatti fare assunzioni sulle politiche di ordinamento e esecuzione dello scheduler, poich\`{e} dipende dalle scelte architetturali del sistema operativo sottostante. Di conseguenza \`{e} necessario sviluppare una strategia di svolgimento della gara che preservi la coerenza della competizione indipendentemente dal comportamento, talvolta non prevedibile, dello scheduler.
\item Il problema precedente potrebbe essere raggirato introducendo l'accumulo di risorse. Ovvero prima di rilasciare il tratto $Tr_N$, il concorrente $Tr_{N+1}$ deve aver gi\`{a} ottenuto l'accesso al tratto $Tr_{N+1}$. In questo modo, per\`{o}, si potrebbe presentare una prospettiva di stallo. Infatti, se il numero di tratti \`{e} minore o uguale al numero di concorrenti, potrebbe verificarsi un'attesa circolare potenzialmente infinita sui tratti della pista. Bisogna quindi assicurarsi che non ci siano le condizioni per il verificarsi di stalli.
\end{enumerate}

\section{Problematiche di distribuzione}
Un sistema che funzioni grazie alla comunicazione remota fra componenti dislocati in diversi nodi nella rete, presenta certamente delle
specifiche problematiche da affrontare. La pi\`{u} critica riguarda di sicuro la robustezza. \`{E} difficile o addiritura utopico sperare
nell'affidabilit\`{a} della rete. La rete presenta sempre dei fault, la cosa importante \`{e} gestirli e non farli propagare in errori. Sicuramente
quindi bisogner\`{a} minimizzare la probabilit\`{a} che un nodo distribuito, perdendo il contatto con il resto del sistema, possa provocare un 
malfunzionamento globale.
\chapter{Soluzione proposta}
'%%%Architettura alto livello%%%
\subsection{Architettura ad alto livello}
Nella seguente sezione verr\`{a} illustrata l'architettura ad alto livello del sistema sviluppato, 
escludendo i dettagli implementativi e legati al linguaggio.
% Componenti del sistema
\subsubsection{Componenti di sistema}
Le principali componenti del sistema sono:
\begin{description}
\item{Competition}
La \emph{Competition} \`{e} l'unit\`{a} atta ad orchestrare l'avvio e la conclusione della corsa. Tale componente, dunque, \`{e} stata concepita per 
offrire le seguenti funzionalit\`{a}:
\begin{itemize}
\item Configurazione parametri di gara:
	\begin{itemize}
		\item numero di giri;
		\item numero di concorrenti;
		\item circuito;
	\end{itemize}
\item Gestione della sessione di iscrizione e accettazione concorrenti (configurati a livello della componente \emph{Box})
\item Avvio delle componenti necessarie al monitoraggio della gara (quali ad esempio \emph{Monitor})
\item Avvio controllato della competizione vera e propria nel momento in cui tutti i prerequisiti di inizio sono soddisfatti, ovvero:
\begin{itemize}
\item la competizione \`{e} stata configurata correttamente;
\item le componenti di controllo e gestione della competizione sono attive e in attesa di comandi;
\item i concorrenti sono stati correttamente registrati alla competizione e in attesa di partire;
\end{itemize}
\end{itemize}
\item{Competitor}
Il \emph{Competitor} \`{e} l'entit\`{a} pensata ad svolgere la gara. \`{E} caratterizzato dalle seguenti sotto-componenti:
\begin{itemize}
\item \textbf{Auto}, ovvero tutte le caratteristiche fisiche legate all'auto, ovvero:
	\begin{itemize}
		\item motore;
		\item capacit\`{a} del serbatoio;
		\item massima accelerazione;
		\item massima velocit\`{a};
		\item gomme montate (mescola, modello, tipo);
		\item livello usura gomme;
		\item livello della benzina nel serbatoio;
	\end{itemize}
\item \textbf{Guidatore}, cio\`{e} le informazioni che descrivono pi\`{u} dettagliatamente il concorrente in gara:
	\begin{itemize}
		\item nome e cognome pilota;
		\item nome scuderia
	\end{itemize}
\item \textbf{Strategia}, ovvero la strategia che sta adottando il pilota, suggerita dai box e dinamica nel corso della gara:
	\begin{itemize}
		\item style di guida, variabile tra conservativo, normale, aggressivo, a seconda dello stato della macchina e delle
			previsioni fatte dai box
		\item numero di lap prima del pit stop
		\item addizionalmente, quando viene fatto un pit stop, la strategia determina anche quali siano le nuove gomme da montare,
		 	la quantit\`{a} di benzina da avere nel serbatoio e il tempo impiegato per il pit stop.
	\end{itemize}
\end{itemize}
Tutte queste informazioni insieme creano quello che viene definito il concorrente di gara. 
Tali informazioni verranno poi usate nel corso della gara per:
	\begin{itemize}
		\item scegliere al momento giusto la miglior traiettoria da seguire, in base alla presenza o meno di altri concorrenti
			nelle vicinanze e alla difficolt\`{a} del tratto;
		\item fornire costantemente aggiornamenti sul suo stato (tramite una parte del modulo \emph{Stats}, informando il computer di bordo
			riguardo a:
			\begin{itemize}
				\item livello di usura gomme;
				\item livello di benzina;
				\item checkpoint attraversato con tempo di arrivo;
				\item insieme al checkpoint verranno aggiornate le informazioni relative a settore e lap;
				\item velocit\`{a} massima raggiunta.
			\end{itemize}
		\item contattare ad ogni giro il box per ottenere una strategia aggiornata;
		\item se suggerito dai box, effettuare un pitstop;
		\item ritirarsi dalla gara una volta che le condizione dell'auto non permettano di poter correre ulteriormente;
		\item banalmente, continuare a correre fino alla fine delle lap prestabilite, dopodich\`{e} fermarsi.
	\end{itemize}
\item{Circuit}
Il circuito è una risorsa finalizzata ad offrire il piano su cui svolgere la competizione. \`{E} condivisa fra tutti i concorrenti in gara e
offre un insieme di funzionalità per poter conoscere le caratteristiche dei vari tratti della pista (compresi i concorrenti presenti
al momento dell'attraversamento). \`{E} composto dalle seguenti sottocomponenti:
\begin{itemize}
\item \textbf{Checkpoint}
\item \textbf{Path}
\item \textbf{}
\end{itemize}
\item{Stats}
\item{Box}
\item{Radio}
\item{Monitor}
\end{description}
 - lista delle componenti con descrizione ad alto livello del loro scopo
 - se necessario fornire un diagramma delle componenti
% Interazione fra le componenti
\subsubsection{Interazione fra le componenti}
 - descrivere ad alto livello l'interazione fra componenti e se necessario aiutarsi con OCR cards
% Strategia adottata per la correttezza temporale
\subsubsection{Strategia adottata per la correttezza temporale}
% Dimostrazione dell'assenza di stallo
\subsubsection{Assenza di stallo}
%%% Architettura in dettaglio %%
\subsection{Architettura in dettaglio}
% Elenco dei task con descrizione
\subsubsection{Risorse attive}
% Elenco risorse condivise con descrizione
\subsubsection{Risorse passive}
\begin{itemize}
\item{Risorse protette}
\item{Altre risorse}
\end{itemize}
%"Analisi della concorrenza"
\subsubsection{Analisi della concorrenza}
	%. analisi dell'interazione risorse e task (senza menzionare la distribuzione)
	%. dimostrazione assenza di racecondition
	%. dimostrazione assenza di starvation
\begin{itemize}
\item{Interazione tra risorse condivise e task}
\item{Assenza di racecondition}
\item{Assenza di starvation}
\end{itemize}
%"Distribuzione"
\subsubsection{Distribuzione}
	%. Elenco risorse distribuite
	%. Interazione risorse distribuite
	%. Misure di fault tolerance
\begin{itemize}
\item{Componenti distribuite}%Con motivazione
\item{Interazione fra le componenti distribuite}
\item{Misure di fault tolerance}
\end{itemize}
% Inizializzazione gara
\subsection{Inizializzazione competizione}
% Stop gara
\subsection{Stop competizione}

\section{Competizione}
\section{Circuito}
\section{Pista}
\section{Box}
\section{Tratto}
\section{Traiettoria}
\section{Concorrente}
\section{Sistema di controllo}
\section{Interfaccia di monitoraggio}
\chapter{Analisi del supporto tecnologico}
% Scelta dei linguaggi 
	% Ada 
	% Java
	% Bash per l'avvio
% Scelta tecnologiche per la distribuzione
% Scelta del middleware

\chapter{Glossario}
\chapter{Bibliografia}
\end{document}
