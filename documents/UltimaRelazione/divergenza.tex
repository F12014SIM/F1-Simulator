\section{Punti di divergenza}
\subsection{Prevedibilità della simulazione}
La nostra volontà, già ripetuta più volte, era quella di avere una simulazione che fosse il più possibile (al 100\% se escludiamo la caduta di un nodo remoto garantendo comunque la gestione di eventuali ritardi) prevedibile e calcolabile a priori da parte di un utente. Diciamo che si è considerata la simulazione come una funzione, la quale avrebbe dovuto dare (su richiesta) lo stesso output fissato l'input.
\subsection{L'esperienza utente}
L’avere una simulazione che rispecchi il passare del tempo sull’orologio dell’utente è una caratteristica che si può aggiungere ad un livello superiore dell’architettura del sistema. Deve essere fornita una simulazione di un sistema reale che produca dati che, una volta analizzati, risultino essere corretti. Per quanto riguarda la presentazione di tali dati ad un utente esterno riteniamo che sia un semplice problema di visualizzazione degli stessi. Si è pensato al simulatore più come uno strumento di analisi. La presentazione dei dati in modo 'realistico' è stato considerato quindi un requisito non mandatorio. Inoltre la nostra idea di soluzione porta a una  correttezza di esecuzione anche velocizzata o senza delay, valorizzando così l’esperienza utente con velocizzazione e rallentamento della gara a livello di gui.
Questo diverso punto di vista rispetto a quanto atteso per la produzione del prototipo ha portato alla prima divergenza, che poi, a cascata, ha influenzato ovviamente tutte le altre scelte progettuali e implementative, ed è questo il primo (e principale) punto di divergenza.
\subsection{Distribuzione}
Provando dopo aver compreso la soluzione proposta a pensare ad una eventuale distribuzione nella stessa ci siamo imbattuti in un secondo problema che nel nostro modo di procedere per lo sviluppo del progetto non ha intaccato il ragionamento. Secondo quanto visto studiando la soluzione proposta introdurre la distribuzione significa in parte introdurre delle componenti di non determinismo fuori dal nostro controllo. Questo è un ulteriore punto di divergenza sempre racchiuso nel primo dove la nostra volontà di garantire pieno determinismo e predicibilità ci ha portato verso una soluzione differente.
                  Per mancata dimestichezza con la tecnologia, non si è riuscito a risolvere il problema con quanto offerto dal linguaggio. Questo ha portato in una direzione in cui, non riuscendo a rispecchiare una soluzione sulle primitive/strutture date, si sono dovute reinventare delle strutture che permettessero di risolvere il problema in base alla progettazione studiata. In questo modo si è potuto avere totale controllo / consapevolezza su quanto si stesse utlilizzando. Questo è un altro punto di divergenza, anche se di natura diverso dal precedente in quanto è dovuto non a una scelta progettuale ma a una mancanza di padronanza degli strumenti a disposizione.
\subsection{Assunzioni}
Altro punto di divergenza che ci ha portato a scelte progettuali diverse è dovuta al fatto che abbiamo volutamente evitato di fare assunzioni sull'ambiente di esecuzione, chiedendo requisiti minimi (installazione di librerie come polyorb e xmlada, ambiente per eseguire ada e una installazione java) senza vincoli di processori o quant'altro. Quindi non sono stati effettuati calcoli riguardanti il numero (approssimato) di cicli di clock per l'esecuzione delle operazioni richieste alle entità attive. Tali operazioni non sono solamente quelle legate all’attraversamento dei tratti del circuito ma anche produzione di dati per la presentazione all'utente finale oltre che variazione a piacimento,e in modo scalabile, di un'eventuale intelligenza artificiale.
\subsection{Conclusioni}
Cercando di trarre le conclusioni per quanto riguarda i punti di divergenza ci si rende conto che la nostra idea iniziale di garantire (a nostro modo di vedere) la massima calcolabilità di quello che succede nel sistema (anche distribuendolo) è stata il punto principale di divergenza che ha poi portato a una serie di decisioni che ci hanno allontanato sempre di più dalla soluzione attesa. Le motivazioni che ci hanno spinto verso tali divergenza verranno analizzate in seguito, certo è che pensando fin da subito ad un sistema che funzionasse su sospensioni assolute tutte le altre scelte per risolvere i problemi sarebbero state una conseguenza e quindi saremmo arrivati alla soluzione attesa. Allo stesso modo le nostre scelte iniziali hanno condizionato tutto il progetto portandolo a divergere rispetto a quanto appreso durante il colloquio.