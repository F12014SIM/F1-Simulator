\section{Analisi critica dei possibili errori (da parte nostra)}
\begin{itemize}
\item worst-case troppo pessimistico (e che non si verificherà mai)
\item visione un po’ troppo java-oriented, ma è nel nostro bagaglio culturale.
\item mancanza di padronanza degli strumenti per risolvere il problema (Ada)
\item dato un problema difficile, senza una adeguata esperienza, di trovare necessariamente la soluzione ottima trovando comunque una corretta e funzionante
\item la presentazione della simulazione non risulta essere scandita da dei tempi reali. L’utente che guardi il proprio orologio vedrà due flussi temporali diversi fra competizione e realtà. Sarebbe possibile tuttavia rielaborare i dati (forniti dal simulatore) in modo da poterli presentare associati ad un tempo realistico. Questo avrebbe tuttavia introdotto un secondo problema. Ovvero l’utente non avrebbe potuto interagire sul sistema facendo affidamento sul tempo (far rientrare il concorrente al box ad un determinato istante t).
\item pensiero di entità attiva indipendente dal linguaggio -> pensiero più java oriented? Si.
            La principale causa che ha portato alle divergenze viste ed analizzate in precedenza è sicuramente la nostra abitudine al paradigma di programmazione e al modello di concorrenza "Java". Il ragionamento sviluppato era ovviamente worst-case driven.  Nel nostro caso il caso pessimo era il più catastrofistico possibile senza assunzioni di controllo sulle entità attive. Presa coscienza della soluzione suggerita al colloquio ci rendiamo conto che usando il modello di concorrenza in Ada lo scenario worst-case risulta essere molto più blando rispetto a quello pensato in fase di progettazione. Dopo aver valutato la bozza di soluzione attesa sicuramente il worst-case reale si riduceva al risveglio di tutti le entità attive nel sistema nello stesso istante, fatto che nell'ambiente di esecuzione di Ada non influisce con ritardi significativi (infatti i tempi utilizzati nella simulazione sono millisecondi, che rispetto a una granularità al miliardesimo di secondo dell’ambiente di esecuzione portano a ritardi non visibili da un osservatore ). All'inizio della fase di progettazione si era ipotizzato di affidarsi ad un tempo di riferimento assoluto. Dopo aver riscontrato i primi potenziali problemi nel sistema concorrente locale ( valutazione del nostro worst-case ) e costruiti possibili scenari distribuiti abbiamo riscontrato dal nostro punto di vista l'incontrollabilità del sistema se gestito con tempi assoluti. La soluzione progettata quindi utilizza tempi relativi, gestibili completamente da noi e portando a una simulazione completamente riproducibile e controllabile, anche con i delay e i problemi connessi alla rete. La base di tutti questi ragionamenti, che ci ha portato a commettere errori di valutazione che si sono poi propagati a cascata, è sicuramente l'assunzione di un worst-case che alla luce di quanto visto durante il colloquio orale è ben distante dalla realtà, in quanto non riproducibile nell'ambiente di esecuzione scelto. L'idea della nostra progettazione comunque era di mantenersi volutamente distanti dal linguaggio (e dall'ambiente scelto) per progettare il sistema e quindi si è arrivati a una reingegnerizzazione di strutture e meccanismi già presenti poi nell'ambiente scelto. La scelta di riprogettare alcuni meccanismi è dovuta quasi certamente al pensiero java oriented che abbiamo come bagaglio culturale. Inoltre certamente l’utilizzo dei tempi relativi gestiti a codice è stata influenzata dalla scarsa padronanza del paradigma di programmazione Ada. Probablimente questo utilizzo di tempi relativi ha compromesso la semplicità del prodotto finale anche se riteniamo che il funzionamento sia corretto e verificabile.
 \item Diciamo quindi che la scelta del livello di non determinismo può avere influenzato, al pari della scelta di una progettazione totalmente ‘language-indipendent’, la direzione presa in fase di progettazione. Ci rendiamo conto che in una soluzione come quella suggerita i ritardi di rete avrebbero potuto essere ‘mappati’ su ritardi esistenti nella realtà. Ma questo non è stato un pensiero fatto in fase di analisi. Non si voleva progettare cercando di giustificare o associare il non-determinismo intrinseco dato dalla rete a un non-determinismo reale. Abbiamo voluto invece raggiungere il massimo determinismo possibile per poi aggiungere eventualmente in un secondo momento delle componenti che permettessero di fornire una fattore “random” alla gara.
 \end{itemize}