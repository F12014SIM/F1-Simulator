\section{Il problema / La soluzione attesa}
\begin{itemize}
	\item Abbiamo un sistema composto da entità concorrenti e risorse condivise, nonchè nodi distribuiti. Come possiamo sfruttare le caratteristiche intrinseche di tale sistema per mapparci una competizione di formula 1? Ovvero identificare ogni entità protagonista di una gara di formula 1 come entità presente in un sistema concorrente e distribuito.
          \item Dalla discussione fatta in corso di colloquio abbiamo appreso che tale sistema sarebbe stato auspicabile che fosse ottenuto sfruttando quanto più possibile le caratteristiche delle primitive offerte da un linguaggio per simulare la realtà. Vale a dire, associando piloti a thread concorrenti, il circuito ad una serie di risorse condivise, la possibilità di sospendere processi per simulare l’attraversamento di un tratto, i box come nodi distribuiti e i problemi di rete come problemi di comunicazione pilota/box. In pratica appoggiarsi a strutture già esistenti (che possano essere magari trovate in un linguaggio specifico), implementando solamente le feature aggiuntive per avere la simulazione richiesta.
          \item A soluzione data, si è visto come progettare un tale sistema in ADA sarebbe risultato efficiente, efficace e più semplice (nonchè verificabile e comprensibile).
          \item Quanto prodotto nel corso dello sviluppo del nostro progetto, per quanto produca risultati corretti e prevedibili, non si può dire essere essenziale e semplice. Il risultato è stato frutto di una serie di ragionamenti mirati a obbiettivi (forse diversi) e, dal lato negativo, da un modello di ragionamento diverso da quello atteso, nonchè volto a trattare il problema dato in modo più generico. Vediamo ora quali avrebbero potuto essere i punti di biforcazione in fase di analisi e progettazione che potrebbero aver portato alla divergenza delle due soluzioni.
\end{itemize}