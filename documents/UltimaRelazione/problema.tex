\section{Il problema}
Senza scendere nei dettagli del progetto didattico (già visti nei documenti presentati precedentemente) il problema si può ridurre ad una mappatura di una competizione di formula uno in un sistema composto da entità concorrenti e risorse condivise, nonchè nodi distribuiti.
Si presentava necessario, quindi, identificare ogni entità protagonista di una gara di formula 1 come entità presente in un sistema concorrente e distribuito e sfruttare le sue caratteristiche intrinseche a vantaggio della simulazione.
Vale a dire, associando piloti a thread concorrenti, il circuito ad una serie di risorse condivise, la possibilità di sospendere processi per simulare l’attraversamento di un tratto, i box come nodi distribuiti e i problemi di rete come problemi di comunicazione pilota/box. 
Dalla discussione fatta in corso di colloquio abbiamo appreso che sarebbe stato auspicabile che un sistema come quello accennato fosse ottenuto sfruttando quanto più possibile le caratteristiche delle primitive offerte da un linguaggio per simulare la realtà. In pratica appoggiarsi a strutture già esistenti ( magari già presenti in un linguaggio specifico ), implementando solamente le feature aggiuntive per ottenere la simulazione richiesta.
A soluzione data, si è visto come progettare un tale sistema in Ada sarebbe risultato efficiente, efficace e più semplice (nonchè verificabile e comprensibile).\\
Quanto prodotto nel corso dello sviluppo del nostro progetto, per quanto produca risultati corretti e prevedibili, non si può dire essere essenziale e semplice. Il risultato è stato frutto di una serie di ragionamenti mirati a obbiettivi forse diversi da quelli attesi e, dal lato negativo, da un modello di ragionamento diverso da quello previsto, volto a trattare il problema dato in modo più generico. Vediamo ora quali avrebbero potuto essere i punti in fase di analisi e progettazione che potrebbero aver portato alla divergenza delle due soluzioni.